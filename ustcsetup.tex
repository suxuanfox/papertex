% !TeX root = ./main.tex

\ustcsetup{
  title                = {随机分析中的几种导数},
  title*               = {On Several Types of Derivatives in Stochastic Analysis},
  author               = {张竣淞},
  author*              = {Zhang Junsong},
  idnum                = {2024233014},
  %speciality           = {数学与应用数学},
  %speciality*          = {Mathematics and Applied Mathematics},
  %supervisor           = {XXX~教授, XXX~教授},
  %supervisor*          = {Prof. XXX, Prof. XXX},
  % practice-supervisor  = {XXX~教授, XXX~教授},  % 专业/工程学位的实践导师
  % practice-supervisor* = {Prof. XXX, Prof. XXX},
  % date                 = {2017-05-01},  % 完成时间,默认为今日
  % department           = {数学科学学院},  % 院系,本科生需要填写
  % student-id           = {PB11001000},  % 学号,本科生需要填写
  % secret-level         = {秘密},     % 绝密|机密|秘密|控阅,注释本行则公开
  % secret-level*        = {Secret},  % Top secret | Highly secret | Secret
  % secret-year          = {10},      % 保密/控阅期限
  % reviewer             = true,      % 声明页显示“评审专家签名”
  %
  % 数学字体
  math-style           = TeX,  % 可选:GB, TeX, ISO
  math-font            = newcm,  % 可选:stix, xits, libertinus
}


% 加载宏包

% 定理类环境宏包
\usepackage{amsthm}
\usepackage{amsmath}

% 插图
\usepackage{graphicx}

% 英文图题、表题
\usepackage{bicaption}

% 三线表
%\usepackage{booktabs}

% 表注
%\usepackage{threeparttable}

% 跨页表格
%\usepackage{longtable}

% 算法
%\usepackage[ruled,linesnumbered]{algorithm2e}

% SI 量和单位
\usepackage{siunitx}
\usepackage{tikz}
\usepackage{tikz-cd}
\usepackage{stmaryrd}
\usepackage{url}

\usepackage{natbib}
\bibliographystyle{ustcthesis-authoryear}


% 参考文献使用 BibLaTeX 宏包
%\usepackage[style=ustcthesis-numeric]{biblatex}
%\usepackage[bibstyle=ustcthesis-numeric,citestyle=ustcthesis-authoryear]{biblatex}
%\usepackage[style=ustcthesis-authoryear]{biblatex}
%\usepackage[style=ustcthesis-bachelor]{biblatex}
% 声明 BibLaTeX 的数据库
%\addbibresource{bib/ref.bib}

% 配置图片的默认目录
\graphicspath{{figures/}}

% 数学命令
\makeatletter
\newcommand\dif{%  % 微分符号
  \mathop{}\!%
  \ifustc@math@style@TeX
    d%
  \else
    \mathrm{d}%
  \fi
}
\makeatother
\newcommand\eu{{\symup{e}}}
\newcommand\iu{{\symup{i}}}
\newcommand\tr{{\symup{tr}}}
%\newcommand\det1{{\symup{det}}}
\newcommand{\nm}[2][]{\left\Vert #2 \right\Vert_{#1}}
\newcommand{\abs}[1]{\lvert #1 \rvert}
\newcommand{\bra}[1]{\left( #1 \right)}
\newcommand{\mr}[1]{\mathscr{#1}}
\newcommand{\mb}[1]{\mathbb{#1}}
\newcommand{\td}[1]{\tilde{#1}}
\newcommand{\pd}[1]{\partial_{#1}}
\newcommand{\ipr}[2]{\langle #1,#2 \rangle}
\newcommand{\tp}[2]{#1 \otimes #2}
\newcommand{\tps}[2]{#1 \otimes\cdots\otimes #2}
\newcommand{\ova}[1]{\overset{#1}{\longrightarrow}}
\newcommand{\dm}[1]{\mathrm{Dom}(#1)}
\newcommand{\sym}[1]{\mathrm{Sym}(#1)}




% 用于写文档的命令
\DeclareRobustCommand\cs[1]{\texttt{\char`\\#1}}
\DeclareRobustCommand\env[1]{\texttt{#1}}
\DeclareRobustCommand\pkg[1]{\textsf{#1}}
\DeclareRobustCommand\file[1]{\nolinkurl{#1}}

% hyperref 宏包在最后调用
\usepackage{xcolor}
\usepackage[backref=page]{hyperref}
\hypersetup{
    colorlinks=true,    % 彩色链接(替代边框)
    linkcolor=violet,     % 文中引用链接颜色
    citecolor=violet,      % 参考文献引用标记颜色
}
