\chapter{Malliavin分析入门}
本章内容主要参考引用自\citep{Nua1}.\par

\section{Hilbert空间相关预备}
由于Malliavin分析理论与Hilbert空间紧密关联,所以先在此陈述一些可能用到的Banach空间,Hilbert空间和线性代数相关内容.本节关于泛函分析
的内容主要取自\citep{Brezis},\citep{kunze}和\citep{resi},线性代数相关内容主要取自\citep{lww}\par
如无特别说明,本章涉及的向量空间(自然包括Hilbert空间)总是实线性的,Hilbert空间总是可分的,即有可数稠密子集.
\subsection{直和与Hilbert和}
\begin{definition}
    设$\mathbb{H}$是一个Hilbert空间.称$\{e_n\}_{n\in \mathbb{N}}\subset \mathbb{H}$是$\mathbb{H}$的一个规范
    正交基(orthonormal basis,有时简记为O.N.B.)\footnote{在有些教材或文献中会使用完备规范正交系(complete orthonormal system)一词来表示线性张成的正交补为零空间的集合,
    而O.N.B.表示有限或无限线性组合张成整个空间的集合,二者是等价的,所以不作区分,具体见\citep{zgq}的1.6节},是指:\par
    (1)\ $\abs{e_n}=1,\forall n\geq 1$且当$n\ne m,\ipr{e_n}{e_m}=0$;\par
    (2)\ $\overline{\text{span}\{e_1,e_2,\cdots\}}=\mathbb{H}$,其中$\overline{\text{span}\{e_1,e_2,\cdots\}}$表示$\{e_n\}_{n\in \mathbb{N}}$所有可能的线性组合.
\end{definition}

\begin{definition}
    称Hilbert空间$\mathbb{H}$的子集$T$为全子集(total set),是指:$\overline{\mathrm{span}(T)}=\mathbb{H}$.
\end{definition}

\begin{proposition}
    设Hilbert空间$\mathbb{H}$可分.则$\mathbb{H}$有规范正交基.
\end{proposition} 

对于任意一族集合,可以定义它们的直积和直和,特别地考虑一族向量空间$(V_i)_{i\in I}$,其直积为
\[
    \Pi_{i\in I}V_i=\{(v_i)_{i\in I}:\forall i,v_i\in V_i\},
\]
加法和数乘定义为逐分量地继承自各分量原来所在的向量空间.直和作为直积的子空间,定义为
\[
    \oplus_{i\in I}V_i=\{(v_i)_{i\in I}\in \Pi_{i\in I}V_i:\text{仅有有限分量非}0\}.
\]
如果所有$V_i=V$,记直积为$V^I$,直和为$V^{\oplus I}$.\par
给定一个一个向量空间$V$,和一族子空间$(V_i)_{i\in I}$,定义这族子空间的和为
\[
    \sum_{i\in I}V_i=\{\sum_{i}v_i:\forall i,v_i\in V_i \text{且仅有有限} i,v_i\ne 0\},
\]
若$\forall i,V_i\cap (\sum_{j\ne i}V_j)=\{0\}$,则称该族子空间的和为直和,记为$\oplus_{i\in I} V_i$.
在Hilbert空间的情形下,在子空间直和的基础上有Hilbert和的概念.
\begin{definition}
    给定Hilbert空间$\mathbb{H}$和一列闭子空间$\{E_n\}_{n\in \mathbb{N}}$.称$\mathbb{H}$是$\{E_n\}$的Hilbert和,是指:\par
    (1)\ $\{E_n\}$两两正交,即:若$n\ne m$,则$\forall v_n\in E_n,\forall v_m\in E_m,\ipr{v_n}{v_m}=0$;\par
    (2)\ $\overline{\text{span}(\cup_{n\geq 1}E_n)}=\mathbb{H}$.
    也记为$\mathbb{H}=\oplus_{n}E_n$
\end{definition}
对任一$\mathbb{H}$的闭子空间$M$,对应的正交投影算子记为$\mathrm{Proj}_M$.在一定意义下,Hilbert和给出了$\mathbb{H}$中元素的
分解或逼近方式,即如下定理.
\begin{theorem}
    设Hilbert空间$\mathbb{H}=\oplus_{n\geq 1}E_n$.给定$u\in \mathbb{H}$,定义$u_n=\mathrm{Proj}_{E_n} u$,以及$S_n=\sum_{i=1   }^{n} u_n$.
    则$\lim_{n\to \infty} S_n=u$且
    \[
        \abs{u}^2=\sum_{n=1}^{\infty}\abs{u_n}^2 
    \]
\end{theorem}
\subsection{张量积}
先从代数角度给出向量空间的张量积.在这一部分,尽管可以给出具体的向量空间构造,但更多关心的是泛性质.首先是线性映射定义的扩展.
\begin{definition}
    设$V_1,\cdots,V_n,W$是向量空间.称映射$M:V_1\times \cdots \times V_n\to W$是$n$重线性映射,是指:$M$对任一分量都是线性的,也即:
    任意$1\leq i\leq n,\forall a,b\in \mathbb{R},\forall \alpha,\beta\in V_i$,都有
\[
    M(\cdots,a\alpha+b\beta,\cdots)=aM(\cdots,\alpha,\cdots)+bM(\cdots,\beta,\cdots).
\]   
特别地,当$W=\mathbb{R}$,称$M$为$n$重线性函数.
\end{definition}

先考虑一种简单的情形,即只有两个向量空间.
\begin{theorem}[张量积的存在性和唯一性]
    给定两个向量空间$V,W$.\par
    (1)\ 存在向量空间$L_u$和双线性映射$B_u:V\times W\to L_u$,使得对所有向量空间$L$和双线性映射$B:V\times W\to L$,都
    存在唯一的线性映射$\phi:L_u\to L$,使得下列图标交换:\par
    \[
        \begin{tikzcd}
        V\times W \arrow[rd,"B"swap] \arrow[r,"B_u"] &L_u \ar[d,"\phi"]\\
        &L
        \end{tikzcd}
    \]\par
    (2)\ 若向量空间和双线性映射$(L_u,B_u),(L'_u,B'_u)$都满足(1)中性质,则存在唯一的线性同构$\td{\phi}:L_u\to L'_u$使得下列图标交换:\par
    \[
        \begin{tikzcd}
            V\times W \arrow[rd,"B'_u"swap] \arrow[r,"B_u"] &L_u \ar[d,"\td{\phi}","~"]\\
            &L'_u
        \end{tikzcd}
    \]
\end{theorem}
\begin{remark}
    我们将$(L_u,B_u)$称为张量积,由于$L_u$在同构意义下唯一,因此记$V\otimes W=L_u$,其中的元素$v\otimes w=B_u(v,w)$,一般也直接称$V\otimes W$为$V,W$的张量积.$V\otimes W$可以具体实现为$V\times W$上
    自由向量空间模去一个等价关系得到的商空间,具体参见\citep{lww}的命题15.1.1.
\end{remark}

对于任意有限个向量空间也可以讨论张量积,一种想法是在$n=2$的基础上归纳定义,下面通过多重线性映射给出直接的定义,但可以证明这与归纳定义实际上是一致的.
\begin{definition}
    给定向量空间$V_1,\cdots,V_n$.存在向量空间$M_u$和$n$重线性映射$C_u:V_1\times \cdots\times V_n\to M_u$,使得对任意$n$重线性映射$C:V_1\times \cdots\times V_n\to M$,都存在
    唯一的线性映射$\phi:M_u\to M$,使得下列图表交换:\par
        \[
            \begin{tikzcd}
            V_1\times \cdots\times V_n \arrow[rd,"C"swap] \arrow[r,"C_u"] &M_u \ar[d,"\phi"]\\
            &M
            \end{tikzcd}
        \]\par
\end{definition}

类似于二元情形,将$(M_u,c_u)$称为$V_1,\cdots,V_n$的张量积,其中$M_u$记为$V_1\otimes \cdots \otimes V_n$.根据以上泛性质,实际上给出了双射
\begin{align*}
    \mathrm{Hom}(V_1\otimes \cdots \otimes V_n;M)&\overset{\sim}{\to} \mathrm{Mul}(V_1\times \cdots\times V_n;M)\\
    \phi&\mapsto \phi\circ C_u
\end{align*}

我们还可以定义多重线性函数的张量积.
\begin{theorem}
    设$T:V_1\times \cdots\times V_k\to \mathbb{R}$和$S:V_{k+1}\times\cdots\times V_{k+l}\to \mathbb{R}$是两个多重线性函数,
    则其张量积定义为如下函数:
    \begin{equation}
        \begin{aligned}
            T\otimes S:V_1\times \cdots\times V_k\times V_{k+1}\times\cdots\times V_{k+l}&\to \mathbb{R}\\
            (v_1,\cdots,v_k,v_{k+1},\cdots,v_{k+l})&\mapsto T(v_1,\cdots,v_k)S(v_{k+1},\cdots,v_{k+l})
        \end{aligned}
    \end{equation}
\end{theorem}
\begin{remark}
    容易验证,$T\otimes S$可以看作$(V_1\times \cdots\times V_k)\times(V_{k+1}\times\cdots\times V_{k+l})$上的双线性函数,
    且满足结合律,即$(T\otimes S)\otimes R=T\otimes (S\otimes R)$.
\end{remark}

我们来说明多元张量积在同构意义下的唯一性,这基于以下命题.
\begin{theorem}
    设$V_1,V_2,V_3$为向量空间.则存在以下同构:\par
    (1)\ (幺约束)
    \begin{equation}
        \begin{array}{c c c c c}
            \mathbb{R}\otimes V_1& \overset{\sim}{\longleftrightarrow}&V_1  &\overset{\sim}{\longleftrightarrow} &V_1\otimes \mathbb{R}\\
            t\otimes v&\longmapsto &tv &\longmapsfrom &v\otimes t
        \end{array}
    \end{equation}\par
    (2)\ (结合约束)
    \begin{equation}
        \begin{array}{c c c c c}
             V_1 \otimes (V_2 \otimes V_3) & \overset{\sim}{\longleftrightarrow} & V_1 \otimes V_2 \otimes V_3 & \overset{\sim}{\longleftrightarrow} & (V_1 \otimes V_2) \otimes V_3 \\
            v_1\otimes (v_2\otimes v_3) & \longmapsto & v_1\otimes v_2\otimes v_3 &\longmapsfrom & (v_1\otimes v_2)\otimes v_3
        \end{array}
    \end{equation}\par
    (3)\ (交换约束)
    \begin{equation}
        \begin{array}{c c c}
            V_1\otimes V_2 &\overset{\sim}{\longleftrightarrow}&V_2\otimes V_1\\
            v_1\otimes v_2 &\longmapsto &v_2\otimes v_1
        \end{array}
    \end{equation}
\end{theorem}
 


下面考虑Hilbert空间的情形.设$H,V$为两个一般的Hilbert空间(不假设是可分的),$(h_\alpha)_{\alpha\in A},(v_\beta)_{\beta\in B}$分别是
$H,V$的规范正交基.定义$H\otimes V$为所有形如$\sum c_{\alpha,\beta} h_\alpha\otimes v_\beta,(c_{\alpha,\beta})\in l^2(A\times B)$的全体,
其中$l^2(A\times B)$表示计数测度下平方可积函数全体,这里的$h_\alpha\otimes v_\beta$暂时只是形式上的记号,无其他实际含义.在$H\otimes V$上定义内积如下:
\[
    \ipr{\sum c_{\alpha,\beta} h_\alpha\otimes v_\beta}{\sum d_{\alpha,\beta} h_\alpha\otimes v_\beta}\coloneq \sum c_{\alpha,\beta}d_{\alpha,\beta}.
\]
从而$H\otimes V$成为一个Hilbert空间,一个规范正交基为$(h_\alpha \otimes v_\beta)_{(\alpha,\beta)\in A\times B}$.易见,若
$H,V$都是可分的,则$H\otimes V$也是可分的.\par
类似于向量空间的张量积,对Hilbert空间的张量积也有泛性质.首先定义映射
\begin{equation}
    \begin{aligned}
        \rho:H\times V&\to H\otimes V\\
        (\sum a_\alpha h_\alpha,\sum b_\beta v_\beta)&\mapsto \sum a_\alpha b_\beta h_\alpha\otimes v_\beta.
    \end{aligned}
\end{equation}
容易验证,$\rho$是一个连续双线性映射.

\begin{theorem}
    给定Hilbert空间$H,V$,以及如上定义的$H\otimes V$和连续双线性映射$\rho$.对任意的Hilbert空间$W$和连续双线性映射$\eta:H\times V\to W$,都存在唯一的
    连续线性映射$T_\eta:H\otimes V\to W$,使得下列图表交换:
    \[
        \begin{tikzcd}
            H\times V \ar[r,"\rho"] \ar[dr,"\eta"] &H\otimes V \ar[d,"T_\eta"]\\
            &W
        \end{tikzcd}
    \]
\end{theorem}

张量积是一个比较抽象的概念,但对于一些特殊的Hilbert空间,可以利用泛性质说明张量积同构于更加具体的空间.以下命题说明两个空间上的平方可积
空间的张量积同构于乘积空间上的平方可积空间.
\begin{theorem}
    设$(\Omega_i,\mathscr{F}_i,\mu_i)$是两个$\sigma$有限的测度空间.则有以下等距同构:
    \begin{align*}
        L^2(\Omega_1,\mathscr{F}_1,\mu_1)\otimes L^2(\Omega_2,\mathscr{F}_2,\mu_2)&\overset{\sim}{\longrightarrow} L^2(\Omega_1\times \Omega_2,\mathscr{F}_1\otimes \mathscr{F}_2,\mu_1\times\mu_2)\\
        f\otimes g &\mapsto ((x,y)\mapsto f(x)g(y)) 
    \end{align*}
\end{theorem}

下面给出可分Hilbert张量积的同构刻画,为此先引入HS算子(Hilbert-Schmidt operator)的概念.
\begin{definition}
    给定可分Hilbert空间$H,V$.称算子$T\in \mathscr{L}(H;V)$是一个HS算子,是指:存在一个规范正交基$(e_n)_{n\in \mathbb{N}}$,使得
    \[
        \sum_{n=1}^{\infty} \nm[V]{Te_n}^2 <\infty.
    \]
    若$T$是HS算子,定义其HS范数为$\nm[\text{HS}]{T}\coloneq (\sum_{n=1}^{\infty} \nm[V]{Te_n}^2)^{\frac{1}{2}}$.
    
\end{definition}

\begin{remark}
    实际上,HS算子$T$的HS范数不依赖于规范正交基的选取,因而是标准的.易见,全体HS算子在通常的加法和数乘之下是一个向量空间,记为$\mathscr{L}_{\mathrm{HS}}(H;V)$或$\mathscr{H}(H;V)$,并
    配备HS范数.
\end{remark}

\begin{remark}
    设$T$是一个HS算子,若
    \[
        \nm[1]{T}\coloneq\sup \sum_{n} \abs{\ipr{Tf_n}{g_n}_V}<\infty,
    \]
    其中上确界是取遍所有$H$的规范正交基$\{f_n\}$和$V$的规范正交基$\{g_n\}$,则称$T$为迹算子,$\nm[1]{T}$称为迹范数.\par
    特别地,若$H=V$,则$\sum_{n} \abs{\ipr{Tf_n}{f_n}}$不依赖于基的选取,$\mathrm{Tr}(T)\coloneq \sum_{n} \abs{\ipr{Tf_n}{f_n}}$称为算子$T$的迹.
\end{remark}


\begin{theorem}
    给定可分Hilbert空间$H,V$,$(e_n),(v_m)$分别是$H,V$的规范正交基.则有以下等距同构:
    \begin{align*}
        H\otimes V &\overset{\sim}{\longrightarrow} \mathscr{L}(H;V)\\
        \sum_{n,m} a_{n,m} h_n\otimes v_m &\longmapsto \sum_{n,m}a_{n,m} \ipr{\cdot}{h_n}v_m
    \end{align*}
\end{theorem}

以上同构已经给出了一种比较具体的视角,我们希望对多元张量积也有上述定理中的同构,从而给出更便于操作和理解的实现.以下观点和构造来自于\citep{resi}和\citep{yanhuang},在本章之后内容中,总是采取这种观点.\par
考虑Hilbert空间$H_1,,H_2$,对任意$h_i\in H_i,i=1,2$,定义$h_1\otimes h_2$为乘积空间$H_1\times H_2$上的双线性函数:
\[
    (h_1\otimes h_2)(x_1,\cdots,x_2)=\Pi_{i=1}^{2}\ipr{h_i}{x_i}_{H_i},x_i\in H_i,
\]
记$\mathscr{E}$为所有形如$h_1\otimes h_2$的线性张成,在其上定义内积:
\[
    \ipr{h_1\otimes h_2}{l_1\otimes  l_2}_{\mathscr{E}}=\Pi_{i=1}^{2}\ipr{h_i}{l_i}_{H_i},
\]
而$(H_1\otimes H_2,\ipr{\cdot}{\cdot}_{H_1\otimes H_2})$则定义为$(\mathscr{E},\ipr{\cdot}{\cdot}_{\mathscr{E}})$的完备化,可以证明
若$\{e^{(i)}_m\}$为$H_i$的规范正交基,则$\{e^{(1)}_{m_1}\otimes e^{(2)}_{m_2}\}$为$H_1\otimes H_2$的规范正交基.对于多元情形则有如下定义.
\begin{definition}
    给定Hilbert空间$H_1,\cdots,H_n$,$\{e^{(i)}_m\}$为$H_i$的规范正交基.对乘积空间$H_1\times \cdots\times H_n$上的任一$n$重线性函数$F$,定义其HS范数为$\nm[\mathrm{HS}]{F}$,
    \[
        \nm[\mathrm{HS}]{F}=\sum_{k_1,\cdots,k_n\in \mathbb{N}}\abs{F(e^{(1)}_{k_1},\cdots,e^{(n)}_{k_n})}^2,
    \]
    $H_1,\cdots,H_n$的张量积即为全体HS范数有限的$n$重线性函数.
\end{definition}
特别地,当$H_1=\cdots=H_n=H$时,记其张量积为$H^{\otimes n}$.\par
这个定义与二元情形是统一的.任取$h\otimes v$,它的HS范数为
\begin{align*}
    \nm[\mathrm{HS}]{h\otimes v}^2&=\sum_{i,j} \abs{(h\otimes v)(e_i^{(1)},e^{(2)}_j)}^2\\
    &=\sum_{i,j} \abs{\ipr{h}{e^{(1)}_i}\ipr{v}{e^{(2)}_j}}^2\\
    &=(\sum_i \ipr{h}{e^{(1)}_i}^2)(\sum_j \ipr{v}{e^{(2)}_j}^2),
\end{align*}
由Parsaval等式知,$\nm[\mathrm{HS}]{h\otimes v}=\nm[H_1]{h}\nm[H_2]{v}$.\par

设$S_n$为$n$阶置换群,设$\sigma\in S_n$,对简单张量定义算子$U_\sigma(\tps{h_1}{h_n})=\tps{h_{\sigma(1)}}{h_{\sigma(n)}}$,并将该算子延拓到全空间,
仍记为$U_\sigma$.然后定义对称化算子
\[
    \mathrm{Sym}=\frac{1}{n!} \sum_{\sigma\in S_n} U_\sigma,
\]
和反对称化算子
\[
    \mathrm{Alt}=\frac{1}{n!} \sum_{\sigma\in S_n} \mathrm{sgn}(\sigma) U_\sigma
\]
算子$\mathrm{Sym}$和$\mathrm{Alt}$的值域中的元素分别称为对称张量和交错张量.记$H^{\odot n}=\mathrm{Sym}(H^{\otimes n}),H^{\wedge n}=\mathrm{Alt}(H^{\otimes n})$,
它们上的内积取为张量积空间上内积的$n!$倍.

\section{Hermite多项式与Wiener混沌分解}
如无特别说明,以下讨论的随机变量都是定义在一个完备的概率空间$(\Omega,\mathscr{F},\mathbb{P})$上.\par

\begin{definition}
    给定Hilbert空间$\mathbb{H}$.称随机过程$W=\{W(h):h\in \mathbb{H}\}$是一个等距Gauss过程(或$\mathbb{H}$上的Gauss过程),是指:\par
    (1)\ $\forall h\in \mathbb{H}$,$\mathbb{E}[W(h)]=0$;\par
    (2)\ $\forall n\in \mathbb{N},h_1,\cdots ,h_n\in \mathbb{H}$,$(W(h_1),\cdots,W(h_n))$的联合分布是多元正态分布;\par
    (3)\ $\forall h,g\in \mathbb{H}$,$\mathbb{E}[W(h)W(g)]=\ipr{h}{g}_{\mathbb{H}}$.
    $W$生成的$\sigma$代数$\sigma(W)$记为$\mathscr{G}$.
\end{definition}

\begin{remark}
    显然,等距Gauss过程中的单个随机变量$W(h)$二阶矩和方差为$\mathbb{E}[W(h)^2]=\nm[\mathbb{H}]{h}^2$,且映射$h\mapsto W(h)$是线性的,也即:任意
    $\lambda,\mu\in \mathbb{R},h,g\in \mathbb{H}$,都有:
    \[
        W(\lambda h+\mu g)=\lambda W(h)+\mu W(g),\mathbb{P}-\mathrm{a.s.}.
    \]\par
\end{remark}
以下命题在后续将经常用到.
\begin{proposition}
    $\{e^{W(h)}:h\in \mathbb{H} \}$是$L^2(\Omega,\mathscr{G},\mathbb{P})$的一个全子集.
\end{proposition}

\begin{definition}
    定义$n$次Hermite多项式为
    \[
        H_n(x)=\frac{(-1)^n}{n!}e^{\frac{x^2}{2}}\frac{\mathrm{d}^n}{\mathrm{d}x^n}e^{-\frac{x^2}{2}},n\geq 1,
    \]
    特别地,规定$H_0(x)=1$.
\end{definition}

\begin{remark}
    令二元函数$F(x,t)=e^{tx-\frac{t^2}{2}}$,对变量$t$做Taylor展开,则
    \[
        F(x,t)=\sum_{n=1}^{\infty}t^n H_n(x),
    \]
    由以上表示容易得到Hermite多项式的几个基本性质:\par
    (1)\ $H'_n(x)=H_{n-1}(x)$;\par
    (2)\ $(n+1)H_{n+1}(x)=xH_n(x)-H_{n-1}(x)$;\par
    (3)\ $H_n(-x)=(-1)^n H_n(x)$;\par
    (4)\ 若$n$为奇数,则$H_n(0)=0$;若$n=2k$为偶数,则$H_n(0)=\frac{(-1)^k}{2^k k!}$;\par
    (5)\ $H_n(x)$的最高次项为$\frac{x^n}{n!}$.\par
    根据以上性质,实际可以归纳计算$H_n(x)$,如$H_1(x)=x,H_2(x)=\frac{1}{2}(x^2-1)$等.
\end{remark}

Hermite多项式不是随意定义的,它和正态分布关系密切,具体体现为以下命题.
\begin{proposition}
    设$(X,Y)$是二元正态的,$\mathbb{E}[X]=\mathbb{E}[Y]=0,\mathbb{E}[X^2]=\mathbb{E}[Y^2]=1$.则任意$n,m\in \mathbb{N}$,
    \[
        \mathbb{E}[H_n(X)H_m(Y)]=\begin{cases}
            0,&n\ne m\\
            \frac{(\mathbb{E}[XY])^n}{n!},&n=m.
        \end{cases}
    \]

\end{proposition}

命题的证明只需考虑$\mathbb{E}[F(X,s)F(Y,t)]$和零均值的多元正态分布矩母函数为$M(t)=\mathrm{exp}({\frac{1}{2}t^T\Sigma t})$,其中$\Sigma$为协方差矩阵.\par
借助Hermite多项式,可以给出$L^2(\Omega,\mathscr{G},\mathbb{P})$的直和分解.首先定义$\mathscr{H}_0$表示常值随机变量全体,$\mathscr{H}_n$为$\{H_n(W(h)):h\in \mathbb{H},\nm[\mathbb{H}]{h}=1\}$生成的
闭线性子空间,$n\geq 1$,由前述命题,当$n\ne m$,$\mathscr{H}_n\perp \mathscr{H}_m$.称$\mathscr{H}_n$为$n$阶Wiener混沌(Wiener chaos),$J_n$为$L^2(\Omega,\mathscr{G},\mathbb{P})$到$\mathscr{H}_n$的正交投影.

\begin{theorem}[Wiener混沌分解]
    \[
        L^2(\Omega,\mathscr{G},\mathbb{P})=\oplus_{n=1}^{\infty} \mathscr{H}_n.
    \]
\end{theorem}

\begin{proof}
    设$X\in L^2(\Omega,\mathscr{G},\mathbb{P})$,任意$n$,$X\perp \mathscr{H}_n$,往证X=0,\par
    注意到多项式$x^n$可以表示为有限个Hermite多项式的线性组合,所以$\mathbb{E}[X(W(h))^n]=0$,从而$\mathbb{E}[Xe^{tW(h)}]=0$,
    $\forall t\in \mathbb{R},h\in \mathbb{H},\nm{h}=1$,这等价于$\mathbb{E}[Xe^{W(h)}]=0,\forall h\in \mathbb{H}$.由于
    $\{e^{W(h)}\}$是$L^2(\Omega,\mathscr{G},\mathbb{P})$的全子集,所以$X=0$.
\end{proof}
根据直和分解的性质,任意$X\in L^2(\Omega,\mathscr{H},\mathbb{P})$,$\sum_{k=1}^{n} J_k X\to X,n\to \infty$且
\[
    \mathbb{E}[X^2]=\sum_{n=1}^{\infty} \mathbb{E}[\abs{J_n X}^2].
\]

记$\mathcal{P}_n^0$为形如$p(W(h_1),\cdots,W(h_n))$的随机变量全体,其中$k\in \mathbb{N}$,$h_1,\cdots,h_n\in \mathbb{H}$,
$p$是一个次数不高于$n$的多元多项式,$\mathcal{P}$表示$\mathcal{P}^0_n$在$L^2$中的闭包,则$\mathcal{P}_n=\mathscr{H}_0\oplus \mathscr{H}_1\oplus\cdots\oplus \mathscr{H}_n$.

下面考虑一个简单的例子.
\begin{example}
    令$(\Omega,\mathscr{F},\mathbb{P})=(\mathbb{R},\mathscr{B}(\mathbb{R}),\gamma)$,其中$\gamma$为标准正态分布,任意$A\in \mathscr{B}(\mathbb{R})$,
    \[
        \gamma(A)=\int_A \frac{1}{\sqrt{2\pi}}e^{-\frac{x^2}{2}}\mathrm{d}x,
    \]
    Hilbert空间$\mathbb{H}$取为$\mathbb{R}$,等距Gauss过程取为$W(h)(x)=x$.注意到$\mathbb{R}$中模长为$1$的只有$1,-1$,且$H_n(x)=(-1)^nH_n(-x)$,所以
    $\mathscr{H}_n$是$H_n(x)$生成的一维线性子空间,这实际上也说明Hermite多项式是$L^2(\mathbb{R},\gamma)$中的一个规范正交基.
\end{example}

设$\mathbb{H}$是无穷维的Hilbert空间,$\{e_i\}$是一个规范正交基.记$\Lambda=\oplus_{n=1}^\infty \mathbb{N}$.对$a\in \Lambda$,定义
$a!=\Pi_{n=1}^\infty (a_i !),\abs{a}=\sum_{n=1}^{\infty} a_i$,根据直和定义,这两个量分别是有限积和有限和,因而是良定的.定义扩展的Hermite多项式为
\[
    H_a(x)=\Pi_{n=1}^\infty H_{a_i} (x_i),x\in \mathbb{R}^{\mathbb{N}}, a\in \Lambda.
\]
以及
\[
    \Phi_a=\sqrt(a!)\Pi_{n=1}^\infty H_{a_i}(W(e_i)),a\in \Lambda.
\]

而$\{\Phi_a:a\in \Lambda\}$是一个规范正交系,即元素都为$1$且两两正交.特别地,$\{\Phi_a:\abs{a}=n\}$还是$\mathscr{H}_n$的一个规范正交基.\par
$\mathbb{H}$上的$n$次对称张量空间$(H^{\odot n},\sqrt{n!}\nm[\mathbb{H}^{\otimes n}]{\cdot})$和$\mathscr{H}_n$间有一个自然的同构:
\[
    \mathrm{Sym}(e^{\otimes a})\mapsto \sqrt{a!}\Phi_a,
\]
其中$e^{\otimes a}\coloneq \otimes_{n=1}^\infty e_i^{\otimes a_i},a=(a_1,a_2,\cdots)\in \Lambda,\abs{a}=n$
\begin{remark}
    对任意Hilbert空间$V$,$L^2(\Omega;V)$也有类似的分解:
    \[
        L^2(\Omega;V)=\oplus_{n=1   }^\infty \mathscr{H}_n(V),
    \]
    其中$\mathscr{H}_n(V)$是由形如$\sum_{i=1}^{n}F_iv_i,F_i\in \mathscr{H}_n.v_i\in V$的$V$值随机变量生成的闭子空间,并且容易得到
    $\mathbb{H}^{\odot n}\otimes V$和$\mathscr{H}_n(V)$之间是同构的.
\end{remark}
\section{Malliavin导数算子及其相关性质}
在本节和下一节中取$\mathscr{F}=\mathscr{G}$.\par

记$C^\infty_p(\mathbb{R}^n)$表示$\mathbb{R}^n$上本身及其各阶导数都只有至多多项式增长的无穷可微函数.记$\mathrm{Srv}(\mathbb{R})$为具有以下形式的
随机变量$F$全体:存在$n\in \mathbb{N},h_1,\cdots,h_n,f\in C^\infty_p(\mathbb{R}^n)$,使得
\[
    F=f(W(h_1),\cdots,W(h_n)).
\]
$\mathrm{Srv}$中的元素称为光滑随机变量.若$f$及其各阶偏导数还是有界的,则记$F\in \mathrm{Srv}_b(\mathbb{R})$;若$f$还有紧支集,则记$F\in \mathrm{Srv}_0(\mathbb{R})$.若$f$是多项式,记$F\in \mathcal{P}$.当只出现$\mathrm{Srv}$时,就是指$\mathrm{Srv}(\mathbb{R})$,
$\mathrm{Srv}_b,\mathrm{Srv}_0$同理.$\mathrm{Srv}_0,\mathcal{P}$都是$L^2(\Omega)$的稠密子集.\par
首先对光滑随机变量定义导数.
\begin{definition}
    设$h_1,\cdots,h_n\in \mathbb{H},f\in C^\infty_p(\mathbb{R}^n)$,$F=f(W(h_1),\cdots,W(h_n))$的导数定义为如下
    $\mathbb{H}$值随机变量:
    \[
        DF=\sum_{i=1}^{n} \partial_{i}f(W(h_1),\cdots,W(h_n)) h_i.
    \]
\end{definition}
\begin{remark}
    同一个光滑随机变量可能有不同的表示形式,最简单的例如$W(h)=\frac{1}{2}W(2h),W(h_1)+W(h_2)=W(h_1+h_2)$,这两个例子展示的表示的不同是由线性性带来的,我们推至一般的情形.
    将$h_1,\cdots,h_n$视为列向量$(h_1,\cdots,h_n)^T$.记$\vec{h}=(h_1,\cdots,h_n)^T$,借用矩阵乘法的记号和规则,$DF$可记为$(\nabla f )\vec{h}$.\par
    假设存在矩阵$A\in \mathbb{R}^{n\times m}$和$\vec{l}=(l_1,\cdots,l_m)$使得$\vec{h}=A\vec{l}$.记$\td{f}=f\circ A\in C^\infty_p(\mathbb{R}^m)$.
    且$F=f(W(h_1),\cdots,W(h_n))=\td{f}(W(l_1),\cdots,W(l_m))$给出了两种不同的表示,但实际上
    \begin{align*}
        (\nabla \td{f}(W(l_1),\cdots,W(l_m)))\vec{l}&=(\nabla f(A(W(l_1,\cdots,W(l_n))))A\vec{l}\\
        &=(\nabla f(W(h_1),\cdots,W(h_n)))\vec{h},
    \end{align*}
    其中第二个等号利用了映射$h\mapsto W(h)$的线性性.
    另外,若对线性无关的$h_1,\cdots,h_n$,有$F=f(W(h_1),\cdots,W(h_n))=g(W(h_1),\cdots,W(h_n))$,则$f=g$,这是因为
    当$h_1,\cdots,h_n$线性无关时,$W(h_1),\cdots,W(h_n)$诱导了非退化的多元正态分布.
\end{remark}

由以上注记,还可以给出光滑随机变量的乘积的导数公式.对$F=f(W(h_1),\cdots,W(h_n)),G=g(W(l_1),\cdots,W(l_k))$,定义$\mathbb{R}^{n+k}$上的函数$\td{f}$和$\td{g}$如下:
\begin{align*}
    &\td{f}(x_1,\cdots,x_n,x_{n+1},\cdots,x_{n+k})=f(x_1,\cdots,x_n)\\
    &\td{g}(x_1,\cdots,x_n,x_{n+1},\cdots,x_{n+k})=g(x_{n+1},\cdots,x_{n+k}),
\end{align*}
则
\[
    FG=\td{f}\td{g}(W(h_1),\cdots,W(h_n),W(l_1),\cdots,W(l_k)),
\]
计算即得$D(FG)=F(DG)+G(DF)$.\par
依定义,对于光滑随机变量$F=f(W(h_1),\cdots,W(h_n))$,导数$DF$是$\mathbb{H}$值的随机变量,自然关心它和$\mathbb{H}$中的元素做内积,对任意$h\in \mathbb{H}$,
\begin{align*}
    \ipr{DF}{h}&=\ipr{\sum_{i=1}^{n} \partial_{i}f(W(h_1),\cdots,W(h_n)) h_i}{h}\\
    &=\sum_{i=1}^{n} \partial_{i}f(W(h_1),\cdots,W(h_n))\ipr{h_i}{h},
\end{align*}
所以$\ipr{DF}{h}$是$f$在$(W(h_1),\cdots,W(h_n))$处沿$(\ipr{h_1}{h}),\cdots,\ipr{h_n}{h})$的方向导数.并且对该方向导数还有以下定理.
\begin{theorem}
    设$F\in \mathrm{Srv}$,$h\in \mathbb{H}$.则
    \[
        \mathbb{E}[\ipr{DF}{h}]=\mathbb{E}[FW(h)].
    \]
\end{theorem}

\begin{proof}
    不妨设$\nm{h}=1$,$F=f(W(h_1),\cdots,W(h_n))$.而$h_1,\cdots,h_n,h$总能通过正交化得到一个极大线性无关的规范正交系,记为$h,e_2,\cdots,e_k$,再由线性变换可得
    $F=g(W(h),W(e_2),\cdots,W(e_k))$,这样,
    \[
        DF=\partial_{1}g(W(h),W(e_2),\cdots,W(e_k))h+\sum_{i=2}^k \partial_{i}g(W(h),W(e_2),\cdots,W(e_k))e_i,
    \]
    从而$\ipr{DF}{h}=\partial_{1}g(W(h),W(e_2),\cdots,W(e_k))$.\par
    记$\phi_k(x)=\frac{1}{(\sqrt{2\pi})^n}e^{-\frac{\abs{x}^2}{2}}$表示$k$元标准正态分布的概率密度函数.则有
    \begin{align*}
        \mathbb{E}[\ipr{DF}{h}]&=\int_{\mathbb{R}^k} \partial_{1}g(x)\phi_k(x)\mathrm{d}x\\
        &=\int_{\mathbb{R}^k} x_1f(x)\phi_k(x)\\
        &=\mathbb{E}[FW(h)].
    \end{align*}
\end{proof}.

用$FG$代替上述定理的$F$,就得到了常用的分部积分公式:
\begin{equation}
    \mathbb{E}[FGW(h)]=\mathbb{E}[F\ipr{DG}{h}]+\mathbb{E}[G\ipr{DF}{h}],F,G\in \mathrm{Srv},h\in \mathbb{H}.
\end{equation}

下面将到导数算子的定义域从Srv延拓到更大的集合上.为此,先介绍以下概念.(见\citep{Yosida}第二章第6节)\par
\begin{definition}
    给定两个Banach空间$X,Y$,在乘积空间上定义范数
    \[
        \nm{(x,y)}=(\nm{x}^2+\nm{y}^2)^\frac{1}{2},
    \]
    则乘积空间在此内积下是一个Banach空间.\par
    给定线性映射$A:X\to Y$,$\Gamma(A)=\{(x,Ax):x\in \mathrm{Dom}(A)\}\subset X\times Y$称为算子$A$的图.\par
    称$A$为闭算子,是指:$\Gamma(A)$是$X\times Y$的线性闭子空间;称$A$为可闭算子,是指:$\Gamma(A)$在$X\times Y$中的闭包$\overline{\Gamma(A)}$是
    某线性算子的图,也即$A$有一个闭延拓,此时,将该延拓记为$\bar{A}$.
\end{definition}
对于闭算子和可闭算子的判别有以下充要条件.
\begin{theorem}
    \ \par
    (1)\ 算子$A$是闭算子,当且仅当以下蕴含关系成立:
    \[
        \left(\{x_n\}\subset \mathrm{Dom}(A),x_n\to x,Ax_n\to y\right)\implies \left(x\in \mathrm{Dom}(A),Ax=y\right).
    \]\par
    (2)\ 算子$A$是可闭的,当且仅当以下蕴含关系成立:
    \[
        \left(\{x_n\}\subset \mathrm{Dom}(A),x_n\to 0,Ax_n\to y\right)\implies\left(y=0\right)
    \]
\end{theorem}

\begin{theorem}
    导数算子$D$是$L^p(\Omega)$到$L^p(\Omega;\mathbb{H})$的可闭算子.
\end{theorem}
\begin{remark}
    在证明之前对该定理做出一点补充说明.由于正态分布有任意阶矩,所以任一$F\in \mathrm{Srv}$,总有$F\in L^p(\Omega),\forall p\geq 1$.
    而其导数$DF$作为一个$\mathbb{H}$值随机变量,$\nm[\mathbb{H}]{DF}$是一个实值随机变量也有$p$阶矩,
    只考虑固定的$p\geq 1$,在该定理之前,导数算子$D$的定义域只有Srv,而该定理说明$D$作为$\mathrm{Srv}\subset L^p(\Omega)$到$L^p(\Omega;\mathbb{H})$的线性算子
    存在一个闭延拓,使之成为一个闭算子,而闭图像定理又说明闭线性算子总是有界的.
\end{remark}
\begin{proof}
    为证$D$为可闭算子,任取$\{F_n\}\subset \mathrm{Srv}$,满足$F_n\overset{L^p}{\to }0$且存在$\eta\in L^(\Omega;\mathbb{H})$使得
    $DF_n\overset{L^p}{\to }\eta$.往证$\eta=0$.\par
    对任意$h\in \mathbb{H}$和光滑随机变量$F$,
    \begin{align*}
        \mathbb{E}[\ipr{\eta}{h}F]&=\lim_{n\to \infty} \mathbb{E}[\ipr{DF_n}{h}F]\\
        &=\lim_{n\to \infty}\mathbb{E}[-F_n\ipr{F}{h}+F_n FW(h)]\\
        &=0,
    \end{align*}
    由光滑随机变量的稠密性可得,$\eta=0$.也就证明了$D$是可闭的.
\end{proof}

对不同的$p\geq 1$,算子$D$的闭延拓的定义域是不同的.对于固定的$p\geq 1$,$D$的闭延拓仍记为$D$,定义域记为$\mathbb{D}^{1,p}$,实际上,$\mathbb{D}^{1,p}$是
Srv$\subset L^P(\Omega)$在范数$\nm[1,p]{\cdot}$下的闭包,其中
\[
    \nm[1,p]{F}\coloneq (\mathbb{E}[\abs{F^p}]+\mathbb{E}[\nm[\mathbb{H}]{DF}^p])^{\frac{1}{p}},
\]
$1$表示导数的阶数,$p$表示算子从哪个空间出发.更具体地说,$\mathbb{D}^{1,p}$是满足以下条件的子空间:若$F_n\overset{L^p}{\to} F$,则$DF_n$在$L^p(\Omega;\mathbb{H})$收敛到某
$\mathbb{H}$值的随机变量.\par
特别地,当$p=2$,$\mathbb{D}^{1,2}$在内积$\ipr{\cdot}{\cdot}_{1,2}$下成为一个Hilbert空间,其中
\[
    \ipr{F}{G}\coloneq \mathbb{E}[FG]+\mathbb{E}[\ipr{DF}{DG}],F,G\in \mathbb{D}^{1,2}.
\]
由此立见$\mathbb{D}^{1,2}$是自反的.实际上,任意$p\in (1,\infty)$,$\mathbb{D}^{1,p}$同构于$L^p(\Omega)\times L^p(\Omega;\mathbb{H})$的一个闭子空间,
由Pettis定理知,$\mathbb{D}^{1,p}$也是自反的.\par
对于光滑随机变量$F=f(W(h_1),\cdots,W(h_n))\in \mathrm{Srv}$,可以定义其高阶Malliavin导数.
\begin{definition}
    给定$F=f(W(h_1),\cdots,W(h_n))\in \mathrm{Srv}$,其$k$阶Malliavin导数定义为如下$\mathbb{H}^{\otimes k}$值随机变量
\[
    D^k F\coloneq \sum_{1\leq i_1,\cdots,i_k\leq n} \frac{\partial^k}{\partial x_{i_1}\cdots \partial x_{i_k}}F(W(h_1),\cdots,W(h_n))h_{i_1}\otimes h_{i_k},
\]    

\end{definition}
\begin{remark}
    在多元微积分中,由于光滑函数偏导算子的可交换性,常使用多重指标简化记号,但由于一般张量不是对称的,所以有必要说明先后次序.
\end{remark}

类似于一阶导数算子,高阶导数算子也是可闭的,记$\mathbb{D}^{k,p}$表示$p$阶导数算子$D^k$在$L^p(\Omega)$中的闭延拓的定义域,也即Srv关于范数
$\nm[k,p]{\cdot}$的完备化,其中
\[
    \nm[k,p]{F}\coloneq (\mathbb{E}[\abs{F}^p]+\sum_{i=1}^{k}\mathbb{E}[\nm[\mathbb{H}^{\otimes i}]{D^i F}])^{\frac{1}{p}}
\]
容易验证$\mathbb{D}^{k+1,p}\subset \mathbb{D}^{k,p}\subset\cdots\subset \mathbb{D}^{1,p}\subset \mathbb{D}^{0,p}=L^p(\Omega)$\footnote{规定$\nm[0,p]{\cdot}=\nm[L^p]{\cdot}$},$\forall p\geq 1$.\par
对于给定的$h\in \mathbb{H}$,定义算子$D_h F\coloneq \ipr{DF}{h}$,这也是$L^p(\Omega)$上的可闭泛函,记$\mathbb{D}_h^p$表示其在$L^p(\Omega)$中的闭延拓
的定义域.

$\mathbb{D}^{1,2}$作为Srv关于范数$\nm[1,2]{\cdot}$的闭包,是比较抽象的,不像光滑随机变量有显式的表达式,所以希望找到一些充分条件
或必要条件来辅助判断一个随机变量是否属于$\mathbb{D}^{1,2}$,也即希望找到平方可积随机变量的可微判别准则.事实上有以下定理给出$L^2(\Omega)$中元素可微的充要条件.
\begin{theorem}[平方可积随机变量可微的充要条件]\label{principle}
    设$F\in L^2(\Omega)$,其Wiener混沌分解为$F=\sum_{n=1}^{\infty} J_n F$.则$F\in \mathbb{D}^{1,2}$,当且仅当
    \begin{equation}
        \sum_{n=1}^{\infty} n \mathbb{E}[\abs{J_n F}^2] <\infty.
    \end{equation}
    此时,$\mathbb{E}[\nm[\mathbb{H}]{DF}^2]=\sum_{n=1}^{\infty} n \mathbb{E}[\abs{J_n F}^2] <\infty$.$D(J_n F)=J_{n-1 }(DF)$.
\end{theorem}
\begin{proof}
    证明需要多次使用空间的完备性和Cauchy列的性质.\par
    先对固定的$n,a\in \Lambda,\abs{a}=n$考虑$\Phi_a=\sqrt{a!}\prod_{i=1}^{\infty} H_{a_i}(W(e_i))$.注意$\Phi_a$只是有限个单变量多项式的乘积,所以
    \[
        D\Phi_a=\sum_{j=1}^{\infty}\big(H_{a_{j-1}}(W(e_j))\prod_{i\ne j}H_{a_i}(W(e_i))\big)e_i,
    \]
    且当$a,b\in \Lambda,a\ne b$,$D\Phi_a,D\Phi_b$在$L^2(\Omega;\mathbb{H})$中是正交的,即$\mathbb{E}[\ipr{D\Phi_a}{D\Phi_b}]=0$.
    另外有限$\{e_1,\cdots,e_l\}$的正交性等价于$W(e_1),\cdots,W(e_l)$的独立性,所以可以计算
    \[
        \mathbb{E}[\nm[\mathbb{H}]{D\Phi_a}^2]=\sum_{j=1}^{\infty}\big(\mathbb{E}[(H_{a_{j-1}}(W(e_j)))^2]\prod_{i\ne j}\mathbb{E}[(H_{a_i}(W(e_i)))^2]\big)=\abs{a}=n.
    \]
    下面说明:若随机变量$G\in \mathscr{H}_n \cap \mathbb{D}^{1,2}$,则导数算子和极限可交换,具体表述如下:
    \[
        DG=\sum_{a\in \Lambda,\abs{a}=n} \ipr{G}{\Phi_a} D\Phi_a.
    \]\par
    由于$\{\Phi_a:a\in \Lambda,\abs{a}=n\}$是可数的,为了便于讨论,重排为$\{\Phi_1,\Phi_2,\cdots\}$,则$G=\sum_{i=1}^{\infty}\ipr{G}{\Phi_i}\Phi_i$.
    记$S_k=\sum_{i=1}^{k} \ipr{G}{\Phi_i}\Phi_i$,则
    \begin{align*}
        &\nm[\mathbb{H}]{\sum_{i=m}^{k} \ipr{G}{\Phi_i}D\Phi_i}^2\\
        \leq &\sum_{i=m}^{k}\ipr{G}{\Phi_k}^2\nm[\mathbb{H}]{D\Phi_i}\\
        =&n\sum_{i=m}^{k}\ipr{G}{\Phi_k}^2
    \end{align*}
    由$\{S_k\}$为Cauchy列,$\{DS_k\}$也是Cauchy列,再结合导数算子的连续性,即得$DG=\sum_{i=1}^{\infty}DS_k$,且$\mathbb{E}[\nm[\mathbb{H}]{DG}^2]=n \mathbb{E}[\abs{G}^2]$.\par
    由以上说明可知,$J_nF$是可微的,且$D(J_n F)\in \mathscr{H}_{n-1}(\mathbb{H})$.\par
    记$T_k=\sum_{i=1}^k D(J_i F)$,
    \begin{align*}
        \mathbb{E}[\nm[\mathbb{H}]{T_k-T_m}^2]&=\mathbb{E}[\nm[\mathbb{H}]{\sum_{i=m+1}^k D(J_i F)}^2]\\
        &\leq \sum_{i=m+1}^{k}\mathbb{E}[\nm[\mathbb{H}]{D(J_i F)}^2]\\
        &=\sum_{i=m+1}^{k} n \mathbb{E}[\abs{J_i F}^2]
    \end{align*}
    由条件可得,$\{T_k\}$是一个$L^2(\Omega;H)$中的Cauchy列,结合闭算子的连续性即得
    \[
        DF=\sum_{n=1}^{\infty}D(J_n F),
    \]以及
    \[
        \mathbb{E}[\nm[\mathbb{H}]{DF}^2]=\sum_{n=1}^{\infty} n \mathbb{E}[\abs{J_n F}^2].
    \]
\end{proof}
\begin{remark}
    以上讨论可以逐次反复进行,从而得到$F\in \mathbb{D}^{k,2}$的充要条件为:
    \[
        \sum_{n=k}^{\infty}\frac{n!}{(n-k)!} \mathbb{E}[\abs{J_n F}^2]<\infty.
    \]
\end{remark}

对于复合函数的Malliavin导数有类似多元微积分的链式法则,具体陈述如下.
\begin{theorem}
    设$p\geq 1$,$\phi:\mathbb{R}^m\to \mathbb{R}$连续可微,且偏导数有界,$F=(F^1,\cdots,F^m),F^i\in \mathbb{D}^{1,p},i=1,\cdots,m$.则$\phi(F)\in \mathbb{D}^{1,p}$
    且
    \[
        D\phi(F)=\sum_{i=1}^{m}\partial_{i}\phi(F) DF^i.
    \]
\end{theorem}

\begin{proof}
    先考虑$\phi:\mathbb{R}\to \mathbb{R}$是光滑函数,$F=f(W(h_1),\cdots,W(h_n))\in \mathrm{Srv}$.此时,$\phi\circ f:\mathbb{R}^n\to R$属于$C^\infty_p$,
    所以$\phi(F)=(\phi\circ f)(W(h_1),\cdots W(h_n))\in \mathrm{Srv}$,且
    \begin{align*}
        D\phi(F)&=\sum_{i=1}^{n} \partial_{i}(\phi\circ f)(W(h_1),\cdots W(h_n))h_i\\
        &=\sum_{i=1}^{n} \phi'(F)\partial_{i}f(W(h_1),\cdots W(h_n))h_i\\
        &=\phi'(F)DF.
    \end{align*}
    设$F\in \mathbb{D}^{1,p}$,由Srv的稠密性,存在一列光滑随机变量$\{F_n\}$使得$F_n$在$\nm[1,p]{\cdot}$下收敛到$F$,也即
    \[
        \nm[1,p]{F_n-F}=(\mathbb{E}[\abs{F_n-F}^p]+\mathbb{E}[\nm[\mathbb{H}]{DF_n-DF}^p])^{1/p}\to 0,
    \]
    由此,$\phi(F_n)\overset{L^p}{\to}\phi$以及$\{D\phi(F_n)\}$是Cauchy列,结合算子$D$的闭性,可得
    \[
        \phi(F)\in \mathbb{D}^{1,p},D\phi(F_n)\overset{L^p}{\to} D\phi(F),
    \]
    \begin{align*}
        &\mathbb{E}[\nm[\mathbb{H}]{D\phi(F)-\phi'(F)DF}^p]\\
        \leq &C_p\big(\mathbb{E}[\nm[\mathbb{H}]{D\phi(F)-D\phi(F_n)}^p]+\mathbb{E}[\nm[\mathbb{H}]{\phi'(F_n)-\phi'(F)DF}^p]\big)\\
        \leq &C_p\big(\mathbb{E}[\nm[\mathbb{H}]{D\phi(F)-D\phi(F_n)}^p]+\mathbb{E}[\nm[\mathbb{H}]{\phi'(F_n)DF_n-\phi'(F_n)DF}^p]+\mathbb{E}[\nm[\mathbb{H}]{\phi'(F_n)DF-\phi'(F)DF}^p]\big)
    \end{align*}
    令$n\to \infty$,结合之前的收敛性,即得$D\phi(F)=\phi'(F)DF$.\par
    去掉$\phi$光滑的假设,考虑函数$\mathscr{p}(x)=Ce^{\frac{1}{\abs{x}^2-1}}I_{(0,1)}(x)$,其中常数$C$使得$\mathscr{p}$在整个实轴上积分为1,以及
    $\mathscr{p}_\epsilon(x)=\frac{1}{\epsilon}\mathscr{p}(\frac{x}{\epsilon})$.则$\mathscr{p}_\epsilon*\phi$是光滑的,
    且导数有界,由以上结论可得
    \[
        D((\mathscr{p}_\epsilon*\phi)(F))=(\mathscr{p}_\epsilon*\phi)'(F)DF=(\mathscr{p}_\epsilon*\phi')(F)DF,
    \]
    由于导数有界,所以$\abs{\phi(x)}=\abs{\phi(0)+\int_{0}^{1}\phi'(tx)x \mathrm{d}t}\leq C(1+\abs{x})$,而磨光子的性质保证了逐点收敛,
    由控制收敛定理可得$(\mathscr{p}_\epsilon*\phi)(F)\overset{L^p}{\to}\phi(F)$,类似可得$(\mathscr{p}_\epsilon*\phi')(F)DF\overset{L^p}{\to }\phi'(F)DF$,最后由算子$D$的闭性可得
    $\phi(F)\in \mathbb{D}^{1,p}$,且$D\phi(F)=\phi'(F)DF$.
    对于$\phi$是多元函数的情形,论证是类似的.
\end{proof}
\begin{remark}
    Malliavin导数的链式法则虽然形式上与经典微积分中的链式法则相似,但还是有一定局限性,例如不能直接取$\phi(x,y)=xy$来计算随机变量乘积的导数,尽管形式上看起来十分合理,实际上一般的随机变量乘积未必是Malliavin可导的,还需要合适的
    可积性条件.
    其中一个原因是,经典微积分中的导数定义是局部的,固定一点处的导数只取决于局部性态,而太远处的行为则影响不大,但Malliavin导数则不然,它依赖于全空间上的积分.\par
    另外,要求导数有界,只是为了是$\phi(F)$和$F$的导数有相同的可积性,实际上可以考虑导数具有多项式增长,依然证$\phi(F)$可导,但要求的导数$\phi'$增长速度越快,$D\phi(F)$可积性越低
\end{remark}
事实上,注意到证明中对$\phi$的关键要求是导数有界,所以对$\phi$的要求可以减弱到仅是Lipschitz连续的.为了证明Lipschitz函数的链式法则,
需要以下引理.
\begin{lemma}
    设$p>1$,$\{F_n\}\subset \mathbb{D}^{1,p}$,$F_n\overset{L^p(\Omega)}{\longrightarrow} F$,且
    \[
        \sup_n \mathbb{E}[\nm[\mathbb{H}]{DF_n}^p]<\infty.
    \]
    则$F\in \mathbb{D}^{1,p}$且$DF_n$弱收敛到$DF$.
\end{lemma}
\begin{proof}
    首先由于$F_n$在 $L^p(\Omega)$中按$L^p$范数收敛,自然也是按范数有界的,而当$p\in (0,\infty)$,$\mathbb{D}^{1,p}$是自反的,所以存在$\{F_n\}$的一个子列
    $\{F_{n_k}\}$弱收敛到$\mathbb{D}^{1,p}$中某元素,暂记为$G$.事实上,$F_{n_k}$作为$L^p(\Omega)$中的点列也弱收敛到$G$,从而$F=G$.也就有了
    $F\in \mathbb{D}^{1,p}$.\par
    对$\{F_n\}$的任意子列应用以上讨论,可得该子列有弱收敛的子列,这也就说明$F_n$在$\mathbb{D}^{1,p}$中弱收敛到$F$,也就证明了结论.
\end{proof}
\begin{theorem}
    设对函数$\phi:\mathbb{R}^m\to \mathbb{R}$,存在常数$K$,使得对所有$x,y\in \mathbb{R}^m$,
    \[
        \abs{\phi(x)-\phi(y)}\leq K\abs{x-y}.
    \]
    $F=(F^1,\cdots,F^m),F^i\in \mathbb{D}^{1,p},i=1,\cdots,m$.\par
    则$\phi(F)\in \mathbb{D}^{1,p}$且存在一个有界的随机向量$G=(G^1,\cdots,G^m)$,使得
    \[
        D\phi(F)=\sum_{i=1}^{m}G_i DF^i
    \]
\end{theorem}
\begin{remark}
    一个Lipschitz函数几乎处处可导,所以定理中的$G_i$几乎可以视为$\partial_{i}\phi(F)$.
\end{remark}
\begin{proof} 为简单起见,依然假设$m=1$.\par
    记$\phi_n=\mathscr{p}_{\frac{1}{n}}*\phi,n\geq 1$,则$\phi_n$是光滑函数,导数$\abs{\phi_n'}\leq K$,且在整个实轴上,
    $\phi(x)$一致收敛到$\phi(x)$.\par
    根据光滑函数的链式法则有$D\phi_n(F)=\phi'_n(F)DF$.\par
    先考虑等式左边.由于$\phi$是Lipschitz函数,所以$\abs{\phi(x)}\leq \abs{\phi(0)}+\abs{\phi(x)-\phi(0)}\leq \abs{\phi(0)}+K\abs{x}$,
    由此可得$\phi_n(F)\ova{L^p} \phi(F)$,而$\phi'_n(F)$的有界性给出了$D\phi_n(F)$矩的一致有界性,由以上引理可得
    $\phi(F)\in \mathbb{D}^{1,2}$,且$D\phi_n(F)$在$L^2(\Omega;H)$中弱收敛到$D\phi(F)$.\par
    再考虑等式右边.导数的有界性给出了弱收敛和弱极限的存在性,将此极限记为$G$,就得到了结论.\par
    多元情形类似.
\end{proof}
下面说明Malliavin导数与经典导数中“一定条件下,若导数为$0$,则函数为常值函数”的平行对应结果.为此,先陈述以下技术性引理.
\begin{lemma}
    随机变量族$\{W(h)G-D_h G:h\in \mathbb{H},G\in \mathrm{Srv}_b\}\cup\{1\}$是$L^2(\Omega)$的全子集.
\end{lemma}
\begin{remark}
    注意到$W(h)G-D_hG$的形式是由分部积分公式自然得到的.
\end{remark}
\begin{theorem}
    设$F\in \mathbb{D}^{1,1}$,$DF=0$.则$F=\mathbb{E}[F]$.
\end{theorem}
\begin{proof}
    若$F\in \mathbb{D}^{1,2}$,由定理\ref{principle}立得结论成立.\par
    取有界光滑函数$\psi_n$满足以下条件:
    \[
        \psi_n(x)=\left\{\begin{aligned}
            x,&x\in [-n,n],\\
            0,&x\notin (-n-1,n+1),\\
            *,&n<\abs{x}<n+1,
        \end{aligned}
        \right.
    \]
    以及光滑随机变量$F_n\ova{\mathbb{D}^{1,1}} F$.对任意$h\in \mathbb{H}$,$G\in \mathrm{Srv}_b$,
    \begin{align*}
        \mathbb{E}[\psi_n(F_m)(W(h)G-D_hG)]=\mathbb{E}[GD_h(\psi_n(F_m))],
    \end{align*}
    令$n\to \infty$并结合上一引理即可.
\end{proof}

作为以上定理的推论,可以给出$\sigma(W)$中的0-1律.
\begin{corollary}
    设$A\in \sigma(W)$.则
    \[
        I_A\in \mathbb{D}^{1,1} \Leftrightarrow \mathbb{P}(A)=0\text{或}1.
    \]
\end{corollary}

下面考虑任意可分Hilbert空间$V$以及取值于$V$的随机变量的Malliavin导数.首先定义$\mathrm{Srv}(V)$为具有以下形式的$V$值随机变量$F$全体:
\[
    F=\sum_{i=1}^{n}F_iv_i,n\in \mathbb{N},F_i\in \mathrm{Srv}(\mathbb{R}),v_i\in V,
\]
对这样的随机变量,定义其$k$阶导数为$\mathbb{H}^{\otimes k}\otimes V$值的随机变量
\[
    D^k F=\sum_{i=1}^{n}(D^k F_i)\otimes v_i,
\]
类似于实值情形,可以证明$D^k$是$L^p(\Omega;V)$到$L^p(\Omega;\mathbb{H}^{\otimes k}\otimes V)$的可闭算子,记其闭延拓的定义域
为$D^{k,p}(V)$,范数$\nm[k,p,V]{\cdot}$定义为:
\[
    \nm[k,p,V]{F}=(\mathbb{E}[\nm[V]{F}^p]+\sum_{j=1}^{k}\mathbb{E}[\nm[\mathbb{H}^{\otimes j}\otimes V]{D^jF}^p])^{\frac{1}{p}}.
\]


\section{导数算子的伴随--散度算子}
\begin{definition}
    称(无界)算子$\delta :L^2(\Omega;\mathbb{H})\supset \mathrm{Dom}(\delta)\to L^2(\Omega)$为导数算子$D$的伴随,是指:\par
    (1)\ $u\in \dm{\delta}$,当且仅当存在常数$C=C(u)$,使得对任意$F\in \mathbb{D}^{1,2}$,
    \[
        \mathbb{E}[\ipr{DF}{u}_{\mathbb{H}}]^2\leq C \mathbb{E}[F^2];
    \]\par
    (2)\ 若$u\in \dm{\delta}$,则$\delta(u)$由
    \[
        \mathbb{E}[F\delta(u)]=\mathbb{E}[\ipr{DF}{u}_{\mathbb{H}}],F\in \mathbb{D}^{1,2}
    \]
    唯一确定.\par
    称$\delta$为(Malliavin)散度算子.
\end{definition}
\begin{remark}
    散度算子是闭线性算子,且易见$\mathbb{E}[\delta(u)]=0,\forall u\in \dm{\delta}$.下面使用记号$\delta(u)$时,总默认$u\in \dm{\delta}$.
\end{remark}

先对$L^2(\Omega;\mathbb{H})$中形式较为简单的元素进行讨论.设$u=\sum_{i=1}^{n}F_ih_i\in \mathrm{Srv}(\mathbb{H})$,
由导数算子的分部积分公式可得
\[
    \delta(u)=\sum_{i=1}^{n}F_iW(h_i)-\sum_{i=1}^{n}\ipr{DF_i}{h_i}_{\mathbb{H}},
\]
实际上,有一些文献也以上式为出发点来定义散度算子.\par
对于$u\in \mathbb{D}^{1,2}(\mathbb{H})$,其导数$Du$是一个$\mathbb{H}\otimes \mathbb{H}$值的随机变量,而$\mathbb{H}\otimes \mathbb{H}$中的元素
可以视为HS算子,其同构由
\[
    e_i\otimes e_j\mapsto \ipr{\cdot}{e_i}e_j
\]
给出.在这种观点下,对$u=\sum_{i=1}^{n}F_ih_i\in \mathrm{Srv}(\mathbb{H})$,
\[
    Du=\sum_{i=1}^{n} DF_i\otimes h_i,
\]
$Du$作为HS算子作用在$h\in \mathbb{H}$则是
\[
    \ipr{Du}{h}=\sum_{i=1}^{n}\ipr{DF_i}{h}_{\mathbb{H}}h_i,
\]
等号左边的$\ipr{Du}{h}$只是泛函分析中“约定”记号,不代表真正的内积,也记$D_hu=\ipr{Du}{h}$.而二阶Malliavin导数也是
$\mathbb{H}\otimes \mathbb{H}$值的随机变量:对一个光滑随机变量$G=g(W(v_1),\cdots,W(v_n))$,
\[
    D^2 G=\sum_{i,j} \partial_{ij}g(W(v_1),\cdots,W(v_n))v_i\otimes v_j,
\]
$D^2G$作为HS算子作用在$h\in \mathbb{H}$上式可记为
\[
    \langle D^2 G,h\rangle=\sum_{i,j}\partial_{ij}g(W(v_1),\cdots,W(v_n))\ipr{h}{v_i}_{\mathbb{H}} v_j,
\]
将其与$l\in \mathbb{H}$做内积可得
\[
    \ipr{\langle D^2 G,h\rangle}{l}_{\mathbb{H}}=\sum_{i,j}\partial_{ij}g(W(v_1),\cdots,W(v_n))\ipr{h}{v_i}_{\mathbb{H}} \ipr{l}{v_j}_{\mathbb{H}},
\]
注意光滑函数二阶偏导的对称性,实际上有$\ipr{\langle D^2 G,h\rangle}{l}_{\mathbb{H}}=\ipr{\langle D^2 G,l\rangle}{h}_{\mathbb{H}}$.
结合以上讨论和分部积分公式计算可得,
\begin{equation}
    D_h(\delta(u))-\delta(D_h u)=\ipr{u}{h}_{\mathbb{H}},
\end{equation}
借用换位子的记号,上式可记为$\ipr{u}{h}_{\mathbb{H}}=[D_h,\delta](u)$,其中$[D_h,\delta]=D_h\circ \delta-\delta \circ D_h$,将上式简称为对易关系.
由该对易关系,可得以下重要结论,它说明了$\dm{\delta}$包含了一类重要的$\mathbb{H}$值随机变量.
\begin{theorem}\label{lab1}
    $\mathbb{D}^{1,2}(\mathbb{H})\subset \dm{\delta}$.若$u,v\in \mathbb{D}^{1,2}(\mathbb{H})$,则
    \[
        \mathbb{E}[\delta(u)\delta(v)]=\mathbb{E}[\ipr{u}{v}_{\mathbb{H}}]+\mathbb{E}[\tr(Du\circ Dv)].
    \]
\end{theorem}
\begin{proof}
    若$u,v\in \mathrm{Srv}(\mathbb{H})$,由分部积分公式和对易关系可计算定理陈述中的等式成立,并且给出了以下估计:
    \[
        \mathbb{E}[\delta(u)^2]\leq \nm[1,2,\mathbb{H}]{u}^2.
    \]\par
    对任一$u\in \mathbb{D}^{1,2}(\mathbb{H})$,存在一列$\{u^n\}\subset \mathrm{Srv}(\mathbb{H})$,使得
    $u^n\ova{L^2(\Omega;\mathbb{H})}u$且$Du^n\ova{L^2(\Omega;\tp{\mathbb{H}}{\mathbb{H}})}Du$,所以由以上估计可得
    $\delta(u^n)$在$L^2(\Omega)$中收敛到某极限,并且可以验证该极限就是$\delta (u)$.
\end{proof}

前述对易关系只是对光滑随机变量而言,实际上可以推至更一般的情形.陈述如下.
\begin{proposition}\label{lab2}
    设$u\in \mathbb{D}^{1,2}$且$D_hu\in \dm{\delta}$.则$\delta(u)\in \mathbb{D}_h^2$,且成立对易关系
    \[
        \ipr{u}{h}_{\mathbb{H}}=[D_h,\delta](u).
    \]
\end{proposition}
该命题可看作以下引理的推论.
\begin{lemma}
    设$G\in L^2(\Omega)$,存在$Y\in L^2(\Omega)$,使得对所有的$F\in \mathbb{D}^{1,2}$,
    \[
        \mathbb{E}[G\delta(hF)]=\mathbb{E}[YF].
    \]
    \par
    则$G\in \mathbb{D}_h^2,D_hG=Y$.
\end{lemma}
\begin{remark}
    引理的证明只需考虑对$Y,G$做Wiener混沌分解.需要注意的是,引理中出现了$\delta(hF)$,但$F$不一定是光滑随机变量,所以暂时无法计算,但下一命题说明
    在一定条件下可以像光滑随机变量那样作运算.
\end{remark}
\begin{proposition}
    设$F\in \mathbb{D}^{1,2},u\in \dm{\delta}$使得$Fu\in L^2(\Omega;\mathbb{H})$.则当$F\delta(u)-\ipr{DF}{u}_{\mathbb{H}}\in L^2(\Omega)$时,
    $Fu\in \dm{\delta}$,且\[
        \delta(Fu)=F\delta(u)-\ipr{DF}{u}_{\mathbb{H}}.
    \]
\end{proposition}

\section{白噪声情形的导数算子和散度算子}
前两节是在一般框架下定义了导数算子和梯度算子,这节集中讨论一种重要模型,即$\mathbb{H}$取为一个测度空间上的平方可积函数的全体.\par
给定一个完备概率空间$(\Omega,\mathscr{F},\mathbb{P})$.设$(T,\mathscr{B},\mu)$是一个无原子的$\sigma$有限的测度空间,$W$为$L^2(T,\mathscr{B},\mu)$上的等距Gauss过程,
此时称$W$为强度为$\mu$的Gauss白噪声,简称白噪声.对任意$f,g\in L^2(T)$,有等距同构
\[
    \mathbb{E}[W(f)W(g)]=\int_{T}fg \mathrm{d}\mu.
\]
若$A\in \mathscr{B},\mu(A)<\infty$,则记$W(A)=W(I_A)$.特别地,若$A,B\in \mathscr{B}$测度有限且$ A\cap B=\emptyset$,则$\mathbb{E}[W(A)W(B)]=0$,即
$W(A),W(B)$为独立的Gauss随机变量.对于一般的$f\in L^2(T)$,也将$W(f)$形式地记为$\int_T f \mathrm{d}W$.另外,当使用记号$W(A)$时,默认$A\in \mathscr{B},\mu(A)<\infty$.
\subsection{多元Wiener积分}
设$m\geq 1$,$\mathscr{B}_0=\{A\in \mathscr{B}:\mu(A)<\infty\}$.首先定义乘积测度空间$(T^m,\mathscr{B}^m,\mu^m)$上平方可积函数$f\in L^2(T^m)=L^2(T^m,\mathscr{B}^m,\mu^m)$的
Wiener积分.类似于Lebsgue积分的做法,先在一类简单但稠密的函数上定义.令$\mathscr{E}_m$为具有以下形式的简单函数$f$的集合
\[
    f(t_1,\cdots,t_m)=\sum_{\sigma\in S_m} a_\sigma I_{A_{\sigma(1)}\times\cdots\times A_{\sigma(m)}}(t_1,\cdots,t_m),
\]
其中$a_\sigma\in \mathbb{R},A_1,\cdots,A_m\in \mathscr{B}_0$两两不交.\par
对于简单函数$f=\sum_{\sigma\in S_m} a_\sigma I_{A_{\sigma(1)}\times\cdots\times A_{\sigma(m)}}$,定义其Wiener积分为
\[
    I_m(f)=\sum_{\sigma\in S_m} a_\sigma W(A_1)\cdots W(A_m)
\]
算子$I_m$有以下性质:\par
(1)\ $I_m$是线性的;\par
(2)\ $I_m(f)=I_m(\mathrm{Sym}(f))$,其中$\mathrm{Sym(f)}$表示函数$f$的对称化:
\[
    \mathrm{Sym}(f)(t_1,\cdots,t_m)=\frac{1}{m!}\sum_{\tau\in S_m} f(t_{\tau(1)},\cdots,t_{\tau(m)});
\]\par
(3)\ 对$f\in \mathscr{E}_m,g\in \mathscr{E}_q$,
\[
    \mathbb{E}[I_m(f)I_q(g)]=\left\{\begin{aligned}
        0,&m\ne q,\\
        m!\int_{T^m} \mathrm{Sym}(f)\mathrm{Sym}(g )\mathrm{d}\mu^m,&m=q
    \end{aligned}
    \right.
\]\par
对以上性质不做证明,仅举一个简单的例子.
\begin{example}
    考虑正半轴上Lebsgue测度.取$A=[0,1],B=[2,3]$,$f=I_{A\times B}+2I_{B\times A}$.首先,对任意的$x\geq 0$,$f(x,x)=0$.\par
    注意到$f(1,2)=1$,而$f(2,1)=2$,说明$f$不是对称的,$f$的对称化$\sym{f}$为
    \begin{align*}
        \sym{f}(x,y)&=\frac{1}{2}(f(x,y)+f(y,x))\\
        &=\frac{1}{2}(I_A(x)I_B(y)+2I_B(x)I_A(y)+I_A(y)I_B(x)+2I_B(y)I_A(x))\\
        &=\frac{3}{2}(I_{A\times B}(x,y)+I_{B\times A}(x,y))
    \end{align*}\par
    $f$的Wiener积分为$I_2(f)=W(A)W(B)+2W(B)W(A)=3W(A)W(B)=I_2(\sym{f})$.
\end{example}
\begin{remark}
    尽管这个例子十分简单,但高维情形基本一致,只是涉及到的组合更复杂.从这个例子也可以看出,简单函数的定义中实际已经
    有了对称化的操作,其不对称性来源于系数,所以简单函数的对称化,实际上就是各加项系数的平均化.
\end{remark}

简单函数在$L^2$中是稠密的,证明中要用到$\mu$无原子的条件.无原子的好处之一是对任意的测度有限集$A$,以及任意的$a\in (0,\mu(A))$,都存在
一个$A$的子集可测且其测度正好为$a$.在正半轴和Lebsgue测度的情形下较为直观.

由Fubini定理和Minkowski定理可得,
\[
    \mathbb{E}[I_m(f)^2]=m!\nm[L^2(T^m)]{\sym{f}}^2\leq \nm[L^2(T^m)]{f}^2.
\]
从而算子$I_m$可以延拓到整个$L^2(T^m)$上.\par
对于$f\in L^2(T^p),g\in L^2(T^q)$,定义其$r$-缩并为$f\otimes_r g\in L^2(p+q-2r)$
\[
    f\otimes_r g(t_1,\cdots,t_{p-r},s_1,\cdots,s_{q-r})=\int_{T^r} f(t_1,\cdots,t_{p-r},x)g(s_1,\cdots,s_{q-r},x)\mu^r(\mathrm{d}x).
\]

记$f\td{\otimes} g=\sym{f\otimes g},f\td{\otimes}_r g=\sym{f\otimes_r g}$,以及$L^2_S(T^m)=\sym{L^2(T^m)}$.\par
借助缩并,可以对任意维中平方可积函数的Wiener积分的乘积表示为若干Wiener积分的线性组合.
\begin{theorem}
    设$f\in L^2_S(T^p),g\in L^2_S(T^q)$.则
    \[
        I_p(f)I_q(g)=\sum_{r=0}^{p\wedge q} \binom{p}{r}\binom{q}{r} I_{p+q-2r}(f\otimes_r g).
    \]
\end{theorem}

上述定理一个重要且常用的特例是取$q=1$,此时有
\[
    I_p(f)I_1(g)=I_{p+1}(f\otimes g)+pI_{p-1}(f\otimes_1 g).
\]

由该公式,可以给出一类特殊的对称函数的递推公式,
\[
    I_{m+1}(f^{\otimes (m+1)})=I_{m}(f^{\otimes m})I_1(f)-(m-1)\nm[L^2(T)]{f}I_{m-1}(f^{\otimes (m-1)})
\]
根据这个递推公式以及归纳法可以得到以下命题.
\begin{proposition}
    设$f\in L^2(T)$且$\nm[L^2]{f}=1$.则
    \[
        H_m(W(f))=\frac{1}{m!}I_m(f^{\otimes m}).
    \]
\end{proposition}

记$\overline{L^2_S(T^m)}$为$L^2(T^m)$中对称函数生成的闭子空间.由以上命题,$n$阶Wiener混沌$\mathscr{H}_n\subset I_n(\overline{L^2_S(T^n)})$,再结合不同阶的Wiener积分的正交性可得
$\mathscr{H}_n=I_n(\overline{L^2_S(T^n)})$.从而有以下定理.
\begin{theorem}
    设$F\in L^2(\Omega,\sigma(W),\mathbb{P})$.则存在$f_n\in L^2_S(T^n),n\geq 1$,使得
    \[
        F=\mathbb{E}[F]+\sum_{n=1}^{\infty} I_n(f_n)
    \]
\end{theorem}
\subsection{作为随机过程的导数}
考虑简单情形$F=W(f)^k,f\in L^2(T)$,则$DF=kW(f)^{k-1}f$,$W(f)$是一个实值随机变量,而$f$是一个$T$上的函数.
$DF$作为$L^2(\Omega;L^2(T))$中的元素是说:将任意$\omega\in \Omega$映为一个平方可积函数$k(W(f)(\omega))^{k-1}f$,该函数在$t\in T$处取值为
$k(W(f)(\omega))^{k-1}f_t$.另一方面,也可以说对任一$t\in T$,由导数确定了一个随机变量$kf_tW(f)^{k-1}$,记为$D_t F$.两种观点只是先后顺序的差异.
对$DF$再求导得$D^2F=k(k-1)W(f)^{k-2}f\otimes f$,这是一个$L^2(T^2)$值随机变量,也就是说需要给定两个参数$t_1,t_2\in T$才能确定一个实值随机变量
$k(k-1)f_{t_1}f_{t_2}W(f)^{k-2}$,记为$D^2_{t_1,t_2}F$,反过来说,随机变量族$\{D^2_{t_1,t_2}F\}$也完全确定了二阶导数$D^2F$.\par
推至一般情形,将$F\in \mathbb{D}^{k,2}$的$k$阶导数
\[
    D^k F=\{D^k_{t_1,\cdots,t_k}F:t_1,\cdots,t_k\in T\}
\]
视为$T^k\times\Omega$上的可测函数.
\begin{example}[[0,1]区间上的布朗运动]\label{经典Wiener空间}
    取$\Omega=C_0([0,1;\mathbb{R}^d])$,$\mathbb{H}=L^2([0,1];\mathbb{R}^d)$.定义
    \[
        H^1=\{x\in \mathrm{AC}([0,1];\mathbb{R}^d):\dot{x}\in \mathbb{H}\}
    \]
    这里的$\dot{x}$表示绝对连续函数的导数.子空间$H^1$被称为Cameron-Martin空间.在$H^1$上定义内积
    \[
        \ipr{x}{y}_{H^1}=\ipr{\dot{x}}{\dot{y}}_{\mathbb{H}},
    \]
    在此内积下,$H^1$是一个Hilbert空间.并且嵌入映射$H^1\hookrightarrow \Omega$是连续的.\par
    为便于讨论,设$d=1$.对任意$t\in[0,1]$,记$W(t)=W(I_{[0,t]})$.考虑光滑随机变量$F=f(W(t_1),\cdots,W(t_n))$,$g\in L^2([0,1])$,
    \begin{align*}
        \ipr{DF}{g}&=\sum_{i=1}^{n}\partial_{i}f(W(t_1),\cdots,W(t_n))\ipr{I_{[0,t_i]}}{g}\\
        &=\sum_{i=1}^{n}\partial_{i}f(W(t_1),\cdots,W(t_n))\int_{0}^{t_i} g(s)\mathrm{d}s,
    \end{align*}
    另一方面,固定$\omega$,记$\td{g}(t)=\int_{0}^{t} g(s)\mathrm{d}s$,
    \begin{align*}
        &\lim_{\epsilon\downarrow 0}\frac{F(x+\epsilon \td{g})-F(x)}{\epsilon}\\
        =&\lim_{\epsilon\downarrow 0}\frac{f(W(t_1)(x+\epsilon\td{g}),\cdots,W(t_n)(x+\epsilon\td{g}))-F(\omega)}{\epsilon}\\
        =&\lim_{\epsilon\downarrow 0}\frac{f(x(t_1)+\epsilon \td{g}(t_1),\cdots,x(t_n)+\epsilon \td{g}(t_n))-f(x(t_1),\cdots,x(t_n))}{\epsilon}\\
        =&\sum_{i=1}^{n} \partial_{i}f(x(t_1),\cdots,x(t_n))\td{g}(t_i)\\
        =&\sum_{i=1}^{n} \partial_{i}f(W(t_1)(x),\cdots,W(t_n)(x))\int_{0}^{t_i} g(s)\mathrm{d}s,
    \end{align*}
    所以$\ipr{DF}{g}(x)=\frac{\mathrm{d}}{\mathrm{d}\epsilon}|_{\epsilon=0}F(x+\epsilon \int_{0}^{\cdot}g(s)\mathrm{d}s)$.
    这个等式左边是在Malliavin导数框架下做内积得到的随机变量,而右边则是将$F$看作样本空间$\Omega$上的泛函,在给定的样本点处,计算$F$的方向导数,
    当然这个方向不是任意的,暂时只是取自$H^1$,但实际上,为了保证一定的合理性,也只能取于$H^1$.
\end{example}

考虑$F\in L^2(\Omega)$有如下表示:
\[
    F=\mathbb{E}[F]+\sum_{n=1}^{\infty} I_n(f_n),f_n\in  L^2_S(T^n).
\]
则导数算子和无穷求和可交换,也即如下定理.
\begin{theorem}
    设$F=\mathbb{E}[F]+\sum_{n=1}^{\infty} I_n(f_n),f_n\in  L^2_S(T^n)\in \mathbb{D}^{1,2}$.则
    $D_t F=\sum_{n=1}^{\infty} nI_{n-1}(f(\cdot,t))$.
\end{theorem}

下面介绍和条件期望相关的性质.对$A\in \mathscr{B}$,记$\mathscr{F}_A=\sigma\big(W(B):B\subset A,\mu(B)<\infty\big)$.首先有以下引理.
\begin{lemma}
    设$F=\mathbb{E}[F]+\sum_{n=1}^{\infty} I_n(f_n),f_n\in  L^2_S(T^n)\in L^2(\Omega)$.则
    \[
        \mathbb{E}(F| \mathscr{F}_A)=\sum_{n=0}^{\infty}I_n(f_n(I_A)^{\otimes n}).\footnote{$I_0=\mathrm{id}_{\mathbb{R}}$}
    \]
\end{lemma}
由以上引理可得以下命题.
\begin{proposition}
    设$F\in \mathbb{D}^{1,2}$,$A\in \mathscr{B}$.则$\mathbb{E}(F| \mathscr{F}_A)\in \mathbb{D}^{1,2}$.且
    \[
        D_t\mathbb{E}(F| \mathscr{F}_A)=\mathbb{E}(D_t F | \mathscr{F}_A)I_A(t).
    \]
\end{proposition}

特别地还有如下推论.
\begin{corollary}\label{lab3}
    设$A\in \mathscr{B}$,$F\in \mathbb{D}^{1,2}$是$\mathscr{F}_A$可测的.则$DF$在$T\times \Omega$上几乎处处为0.
\end{corollary}
\subsection{Skorohod随机积分}
在白噪声情形下,散度算子也称为Skorohod随机积分.对$u\in \dm{\delta}$,将$\delta(u)$形式地记为$\int_T u \mathrm{d}W$或$\int_T u_t \mathrm{d}W_t$.
$u\in L^2(\Omega;L^2(T))\ova{\sim} L^2(T\times \Omega)$,对任意$t\in T$,$u_t$作为一个平方可积的随机变量有Wiener混沌分解:
\[
    u_t=\sum_{n=0}^{\infty}I_{n}(f_n(\cdot,t)),
\]
其中$f_n\in L^2(T^{n+1}),n\geq 1$对前$n$个变量是对称的,但不必对所有变量对称.在如上分解的基础上,可以给出$u\in \dm{}$
\begin{proposition}
    设$(u_t)=\big(\sum_{n=0}^{\infty}I_{n+1}(f_n(\cdot,t))\big)\in L^2(T\times \Omega)$.则$u\in \dm{\delta}$,当且仅当级数
\[
    \sum_{n=0}^{\infty} I_{n+1}(\sym{f_n}).
\]
在$L^2(\Omega)$中收敛.
\end{proposition}

\begin{proof}
    设$g\in L^2_S(T^n)$.计算可得
    \begin{align*}
        \mathbb{E}[\ipr{u}{DG}]&=\mathbb{E}[\int_T u_t D_tG\mu(\mathrm{d}t)]\\
        &=\mathbb{E}[\int_T u_t(nI_{n-1}(g(\cdot,t)))\mu(\mathrm{d}t)]\\
        &=\mathbb{E}[\int_T (\sum_{m=0}^{\infty}I_{m}(f_m(\cdot,t)))(nI_{n-1}(g(\cdot,t)))\mu(\mathrm{d}t)]\\
        &=n\int_T \mathbb{E}[I_{n-1}(f_{n-1}(\cdot,t))I_{n-1}(g(\cdot,t))]\mu(\mathrm{d}t)\\
        &=n(n-1)!\int_T \ipr{I_{n-1}(f_{n-1}(\cdot,t))}{I_{n-1}(g(\cdot,t))}\mu(\mathrm{d}t)\\
        &=\mathbb{E}[I_{n-1}(\sym{f_{n-1}})Y].
    \end{align*}\par
    若$u\in \dm{\delta}$,则上式蕴含对任意$G\in \mathscr{H}_n$,
    \[
        \mathbb{E}[\delta(u)Y]=\mathbb{E}[\ipr{u}{DY}]=\mathbb{E}[I_{n-1}(\sym{f}_{n-1})Y],
    \]
    从而说明$J_n(\delta(u))=I_{n-1}(\sym{f}_{n-1})$,进而级数
    \[
        \sum_{n=0}^{\infty} I_{n+1}(\sym{f}_n)
    \]
    收敛且在$L^2(\Omega)$中收敛到$\delta(u)$.\par
    反过来,若级数收敛,记其收敛到$Z$,则
    \[
        \mathbb{E}[\ipr{u}{DY}]=\mathbb{E}[ZY],\forall Y\in \mathscr{H}_n,
    \]
    线性性保证了对$\forall Y\in \oplus_{n=1}^N \mathscr{H}_n$同样有上式成立,再由连续性即得$Z=\delta(u)$.
\end{proof}

记$\mathbb{L}^{1,2}=\mathbb{D}^{1,2}(L^2(T))$,则$\mathbb{L}^{1,2}$同构于$L^2(T;\mathbb{D}^{1,2})$.依定理\ref{lab1},
$\mathbb{L}^{1,2}\subset \dm{\delta}$.若$u,v\in \mathbb{L}^{1,2}$,则\ref{lab1}中的等式在白噪声情形下的对应为
\[
    \mathbb{E}[\delta(u)\delta(v)]=\int_T \mathbb{E}[u_tv_t]\mu(\mathrm{d}t)+\int_T\int_T \mathbb{E}[(D_su_t) (D_tu_s)]\mu(\mathrm{d}t)\mu(\mathrm{d}s).
\]


在白噪声情形下,对易关系\ref{lab2}表述如下.
\begin{theorem}
    设$u\in \mathbb{L}^{1,2}$.对几乎处处$t\in T$,$\{D_tu_s:s\in T\}\in \dm{\delta}$且存在一个版本使得$\{\int_T D_t u_s \mathrm{d}W_s:t\in T\}\in L^2(T\Omega)$.则
    $\delta(u)\in \mathbb{D}^{1,2}$且
    \[
        D_t(\delta(u))=u_t+\delta(D_t u)=u_t+\int_T D_tu_s \mathrm{d}W_s.
    \]
\end{theorem}

\subsection{It\^{o}随机积分}

下面将It\^{o}随机积分作为Skorohod随机积分的特例来讨论.\par
取$T=(0,\infty)$,$\mathscr{B}$为Lebsgue可测集全体,$\mu$为Lebsgue测度.令$B_t=W(t)=W(I_{(0,t]})$为布朗运动.$(\mathscr{F}_t)$为布朗运动生成的自然$\sigma$代数流,并注意到
$\mathscr{F}_t=\mathscr{F}_{(0,t]}$.\par
对适应过程$u\in \mathbb{L}^{1,2}$,$u_t$是$\mathscr{F}_t$可测的,所以由\ref{lab3}可得$D_su_t=0,s>t$,所以对易关系蕴含It\^{o}等距:
\[
    \mathbb{E}[\delta(u)^2]=\mathbb{E}[\int_0^\infty u_t^2 \mathrm{d}t].
\]
考虑具有如下形式的初等随机过程$u$:
\[
    u=\sum_{i=1}^n X_iI_{(t_{i},t_{i+1}]},
\]
其中$0\leq t_0 < t_1 < \cdots <t_n$,$X_i\in L^2(\Omega,\mathscr{F}_{t_i})$,可得
$\delta(u)=\sum_{i=1}^{n}X_i(B_{t_{i}}-B_{t_{i-1}})$. 由标准的逼近方法和散度算子的闭性可得:全体初等随机过程在$L^2((0,\infty)\times \Omega)$中的闭包,记为$L^2_{\mathscr{F}}((0,\infty)\times \Omega)$,中的任一元素
$v$都是适应的,且$v\in \dm{\delta}$.
下面给出扩散过程的一个重要结论:对随机积分$X_t=\int_0^t u_s \mathrm{d}B_s$,其Malliavin可微性取决于“被积函数”$u_s$的可微性,反过来$X_t$的可微性也蕴含了$u_s$的可微性.

在此之前,先对适应性做出一点简单的说明.考虑$F=f(W(t_1),\cdots,W(t_n)$是$\mathscr{F}_T$可测的,$t_1\leq\cdots t_n\leq t$.
则$DF=\sum_{i=1}^{n}\partial_{i}f(W(t_1),\cdots,W(t_n))I_{[0,t_i]}$是一个初等随机过程,依\ref{lab3}有$D_s F=0,s\geq t$.而当$s<t$时,
$D_t F=\sum_{i=1}^{n}\partial_{i}f(W(t_1),\cdots,W(t_n))I_{[0,t_i]}(t)$.定义初等随机过程$u=FI_{(S,R]},S\geq T$.则
$D_t u_s=\sum_{i=1}^{n}\partial_{i}f(W(t_1),\cdots,W(t_n))I_{[0,t_i]}(t)I_{(S,R]}(s)$.固定$t$,考虑$s>t$.只有在$s>S\geq t_n$时,
$D_t u_s\ne 0$才可能成立,此时一定有$D_t u_s\mathscr{F}_s$,从而$\{D_t u_s:s\geq t\}$是适应的.由标准的逼近讨论可知对任一$u\in L^2_{\mathscr{F} }$,
$\{D_tu_s:s\geq t\}$都是适应的.
\begin{theorem}
    设$u\in L^2_{\mathscr{F}}((0,T)\times \Omega)$,$X=\int_{0}^{T}u_s \mathrm{d}B_s$.则$X\in \mathbb{D}^{1,2}$,当且仅当$u\in \mathbb{L}^{1,2}$.\par
    此时,$\{D_tu_s:t\in (0,T)\}\in L^2_{\mathscr{F}}$且
    \[
        D_tX=u(t)+\int_{t}^{T} D_tu_s \mathrm{d}B_s.
    \]
\end{theorem}
\begin{proof}
    若$u\in L^2_{\mathscr{F}}$,则$D_t u_s$几乎处处存在且平方可积,且$\{D_t u_s:s\leq t\}$是适应和平方可积的,所以$D_t u_\cdot\in \dm{\delta}$,由对易关系可得,
    \[
        D_t X=u_t+\int_t^T D_t u_s \mathrm{d}B_s.
    \]\par
    反过来,若$X\in \mathbb{D}^{1,2}$.记$u^n_t$为$u_t$在$\mathcal{P}_n=\mathscr{H}_0\oplus\cdots\oplus \mathscr{H}_n$上的正交投影.令
    $X^n=\int_0^T u_t^n \mathrm{d}B_s$,则$X^n$恰是$X$在$\mathcal{P}_{n+1}$上的正交投影,且在$\mathbb{D}^{1,2}$中收敛到$X$.从而由以下估计,
    \begin{align*}
        \mathbb{E}[\nm[L^2((0,T)^2)]{Du^n}]&=\mathbb{E}[\int_{0}^{T}\int_{0}^{s}\abs{D_t u^n_s}^2 \mathrm{d}t \mathrm{d}s]\\
        &\leq \mathbb{E}[\int_{0}^{T}\int_{0}^{s}\abs{D_t u^n_s}^2 \mathrm{d}t \mathrm{d}s]+\mathbb{E}[\int_{0}^{t} \abs{u_t}^2 \mathrm{d}t]\\
        &\leq \mathbb{E}[\int_{0}^{T} \abs{D_tX^n}\mathrm{d}t].
    \end{align*}
    由该有界性结合引理6.17即可得$u\in \mathbb{L}^{1,2}$.

\end{proof}

下面的定理说明如何已知$X$计算被积函数$u_s$.
\begin{theorem}[Clark-Ocone公式]
    设$X\in \mathbb{D}^{1,2}$.则
    \[
        X=\mathbb{E}[X]+\int_{0}^{t} \mathbb{E}(D_s X|\mathscr{F}_s)\mathrm{d}B_s.
    \]
\end{theorem}
















\section{Ornstein-Uhlenbeck半群}
Ornstein-Uhlenbeck半群相关内容十分丰富,在此仅介绍一些重要的性质.具体内容和证明参见\citep{Nua1}.
\begin{definition}
    定义算子$T_t,t\geq 0$为
    \[
        T_t(F)\coloneq \sum_{n=0}^{\infty} e^{-nt}J_n F,F\in L^2(\Omega).
    \]
    $T_t$形成的单参数算子半群$\{T_t:t\geq 0\}$称为Ornstein-Uhlenbeck半群,简称为OU半群.\par
    记$L$为OU半群的无穷小生成元,也即:
    \[
        LF=\lim_{t\downarrow 0}\frac{T_tF-F}{t},
    \]
    其中$F\in \dm{L}=\{F\in \mathbb{D}^{1,2}:\text{极限}\lim_{t\downarrow 0}\frac{T_tF-F}{t}\text{在强收敛意义下存在}\}$.
\end{definition}

\begin{remark}
    OU半群的无穷小生成元$L$有相对具体的表达式:$LF=\sum_{n=0}^{\infty}-nJ_nF$,并且$L$具有一定的对称性,即
    \[
        \mathbb{E}[G(LF)]=\mathbb{E}[F(LG)],L,G\in \dm{L}.
    \]
\end{remark}

算子$L$和导数算子,散度算子有紧密关联,具体体现在如下命题,这一命题也为实际的计算和应用提供许多便利.
\begin{proposition}
    设$F\in L^2(\Omega)$.则$F\in \dm{L}$,当且仅当$F\in \mathbb{D}^{1,2}$且$DF\in \dm{\delta}$,此时
    \[
        \delta(DF)=-LF.
    \]
\end{proposition}

作为以上命题的应用,可以对$F=f(W(h_1),\cdots,W(h_n))\in \mathrm{Srv}$计算
\[
    LF=\sum_{i,j}\partial_{ij}f(W(h_1),\cdots,W(h_n))\ipr{h_i}{h_j}_{\mathbb{H}}-\sum_{i=1}^{n}\partial_{i}f(W(h_1),\cdots,W(h_n))W(h_i)
\]
更一般地,类似于链式法则有如下定理.
\begin{theorem}
    设$F=(F^1,\cdots,F^m),F_i\in \mathbb{D}^{2,4},i=1,\cdots,m$,$\phi\in C^2(\mathbb{R}^m)$且各一阶,二阶偏导数有界.
    则$\phi(F)\in \dm{L}$,
    \[
        L(\phi(F))=\sum_{i,j} \partial_{ij}\phi(F)\ipr{DF^i}{DF^j}_{\mathbb{H}}+\sum_{i}\partial_{i}\phi(F)LF^i
    \]
\end{theorem}

OU半群最重要的性质之一即超压缩性.
\begin{theorem}[超压缩性]
    设$p>1$,定义$q(t)=(p-1)e^{2t}+1,t>0$.则对任意$F\in L^p(\Omega)$,
    \[
        \nm[q(t)]{T_tF}\leq \nm[p]{F}
    \]
\end{theorem}
\begin{remark}
    称为超压缩性的原因之一是$q(t)$严格大于$p$,而高阶矩可以控制低阶矩.
\end{remark}



















\section{具有Lipschitz系数的SDE的Malliavin可微性} 
本章最后介绍随机微分方程解的Malliavin可微性.考虑方程
\begin{equation}\label{sde1}
    \mathrm{d}X_t=b_t(X_t)\mathrm{d}t+\sigma_t(X_t)\mathrm{d}W_t,X_0=x_0,t\in [0,T]
\end{equation}
其中$(W_t)_{t\geq 0}=(W^1_t,\cdots,W^d_t)$是定义在一个完备带流概率空间$(\Omega,\mathscr{F},(\mathscr{F}_t)_{t\geq 0},\mathbb{P})$上的$d$维布朗运动.
\begin{align*}
    &b:[0,T]\times \mathbb{R}^m\to \mathbb{R}^m\\
    &\sigma:[0,T]\times \mathbb{R}^m\to \mathbb{R}^{m\times d}
\end{align*}
为可测映射,分别称为漂移系数和扩散系数.记$b=(b^1,\cdots,b^m)^T$,$\sigma=(\sigma^{i,j})_{m\times d}$作为一个矩阵值映射,记其第$j$列为$A^j=(\sigma^{j,1},\cdots,\sigma^{j,m})^T$,所以方程\ref{sde1}还可以写为

\begin{equation}\label{sde2}
        X_t=x_0+\int_{0}^t b_s(X_s)\mathrm{d}s+\sum_{j=1}^{d}\int_{0}^{t}A^j_t \mathrm{d}W^j_t,t\in [0,T].
\end{equation}

本节讨论的随机微分方程总假设漂移系数和扩散系数满足Lipschitz连续性和有界性假设,简记为方程\ref{sde1}或\ref{sde2}满足假设(A),其中
假设(A)具体如下.
\begin{proof}[A]
    存在常数$K>0$使得\par
(A1)\ $\forall x,y\in \mathbb{R}^m,\forall t\in [0,T]$,
\[
    \abs{b_t(x)-b_t(y)}+\sum_{j=1}^{d}\abs{A^j_t(x)-A^j_t(y)}\leq K\abs{x-y}.
\]\par
(A2)\ 映射$t\mapsto A^j_t(0),t\mapsto b_t(0)$是$[0,T]$上的有界映射.
\end{proof}

若方程\ref{sde1}满足假设(A),则\ref{sde1}有唯一强解,记该唯一强解为$(X_t)_{t\in [0,T]}=(X^1_t,\cdots,X^m_t)_{t\in [0,T]}$,具体可见\citep{legall}.\par

与第二节中的记号对应,取$\Omega=C_0([0,T];\mathbb{R}^d)$为$[0,1]$上在$0$处取值为$0$的$\mathbb{R}^d$值连续函数全体,$\mathbb{P}$为Wiener测度,
$\mathscr{F}$为$\mathscr{B}(\Omega)$关于$\mathbb{P}$的完备化,$\mathscr{F}_t$为布朗运动产生的自然流,$\mathbb{H}$为$L^2([0,T];\mathbb{R}^d)$.
另记$\mathbb{D}^{1,\infty}=\cap_{p\geq 1}\mathbb{D}^{1,p}$.
先陈述本节的主要定理如下.
\begin{theorem}[SDE解的Malliavin可微性]\label{SDE解的Malliavin可微性}
    对任意$t\in [0,T],i=1,\cdots,m$,$X^i_t\in \mathbb{D}^{1,\infty}$.进一步地,
    \[
        \sup_{0\leq r\leq t}\mathbb{E}[\sup_{r\leq s\leq T}\abs{D^j_rX^i_s}^p]<\infty,
    \]
    且存在一致有界的$m$维适应过程$\bar{A}^{k,l}_s,\bar{B}^k_s,k\in\{1,\cdots,m\},l\in\{1,\cdots,d\}$,使得
    \[
        D^j_r X_t=A^{j}_r(X_r)+\sum_{k=1}^{m}\sum_{l=1}^d\int_{r}^{t}\bar{A}^{k,l}_sD_r^jX^k_s \mathrm{d}W_s^l+\sum_{k=1}^{m}\int_{r }^{t }\bar{B}^k_s D^j_rX^k_s \mathrm{d}s,r\leq t ,
    \]
    以及$D^j_r X_t,r>t$.
\end{theorem}

在证明该定理之前,先介绍一个技术性引理,具体证明见\citep{Nua1}的命题1.5.5.
\begin{lemma}\label{lab4}
    设$p>\alpha>1$,$F\in \mathbb{D}^{1,\alpha}$.若$DF\in L^p(\Omega;\mathbb{H})$,则$F\in L^p(\Omega)$.
\end{lemma}


\begin{proof}[定理\ref{SDE解的Malliavin可微性}的证明]
    考虑Picard迭代,令$X_0(t)\equiv x_0$,归纳定义
   \[
     X_{n+1}(t)=x_0+\int_{0}^{t}b(s,X_n(s))\mathrm{d}s+\sum_{j=1}^{d}\int_{0}^{t}A^j(s,X_n(s))\mathrm{d}W_s,
   \]记$\psi_n(t)=\sup_{0\leq r\leq t}\mathbb{E}[\sup_{r\leq s\leq T}\abs{D_rX_n(s)}^p],p\geq 2$.\par
   下面使用归纳法证明$X^k_n(t)\in \mathbb{D}^{1,\infty},i=1,\cdots,m,t\in [0,T]$,$\psi_n(t)<\infty$以及存在常数$c_1,c_2$使得
   \[
    \psi_{n+1}(t)\leq c_1+c_2\int_{0}^{t} \psi_n(s)\mathrm{d}s.
   \]
   当$n=0$时显然成立,假设对$k=1,\cdots,n$都有以上结论成立,考虑$k=n+1$.\par
   首先由于系数的Lipschitz连续性,固定$s$,对$\sigma^{i,j}_s,b^i_s,$应用链式法则,可得存在一致有界的随机变量$\bar{A}^{n,i,j,l}_s,\bar{B}^{n,i,l}_s$使得 
   \begin{align*}
    &D\sigma^{i,j}_s(X_n(s))=\sum_{l=1}^{m}\bar{A}^{n,i,j,l}_sDX_n^i(s),\\
    &Db^i_s(X_n(s))=\sum_{l=1}^{n}\bar{B}^{n,i,l}_s DX_n^i(s)
   \end{align*}
   根据归纳假设,$DX_n^i(s)\in \mathbb{D}^{1,\infty}$,而系数又是一致有界的,所以$D\sigma^{i,j}_s(X_n(s)),Db^i_s(X_n(s))\in \mathbb{D}^{1,\infty}$.\par
   由于“被积函数”$D\sigma^{i,j}_s(X_n(s))\in \mathbb{D}^{1,2}$,所以It\^{o}随机积分
   \[
    \sum_{j=1}^{m}\int_{0}^{t} \sigma^{i,j}_s(X_n(s)) \mathrm{d}W_s^j\in \mathbb{D}^{1,2},
   \]
   且
   \[
    D^k_r\left(\sum_{j=1}^{m}\int_{0}^{t} \sigma^{i,j}_s(X_n(s))\mathrm{d}W^j_s\right)=\sigma^{i,k}_r(X_n(r))+\sum_{j=1}^{m}\int_{r}^{t}D^k_r(\sigma^{i,j}_s(X_n(s)))\mathrm{d}W_s^j .
   \]
   对于上面等式右边,由Lipchitz连续性可知,$\sigma^{i,k}_r(X_n(r))$和$X_n(r)$有相同的可积性,对于积分项的控制则由BDG不等式和$\psi_n(t)$的有界性得到:
   \begin{align*}
        &\mathbb{E}[\abs{\int_{r}^{t}D^k_r(\sigma^{i,j}_s(X_n(s)))\mathrm{d}W_s^j}^p]\\
        \leq& \mathbb{E}[(\abs{\int_{r}^{t}(D^k_r(\sigma^{i,j}_s(X_n(s))))^2\mathrm{d}s})^{p/2}]
   \end{align*}
   结合引理\ref{lab4},可得
   \[
    \sum_{j=1}^{m}\int_{0}^{t} \sigma^{i,j}_s(X_n(s)) \mathrm{d}W_s^j\in \mathbb{D}^{1,\infty}.
   \]\par

   对于确定性积分部分,可以将其看作级数的极限,由于导数算子的闭性,$\int_{0}^{t} b^i_s(X_n(s))\mathrm{d}s\in \mathbb{D}^{1,2}$,以及

   \[
    D^k_r\left(\int_{0}^{t} b^i_s(X_n(s))\mathrm{d}s\right)=\int_{0}^{t} D^k_r b^i_s(X_n(s))\mathrm{d}s.
   \]\par

   下面估计$\mathbb{E}[\sup_{r\leq s\leq T}\abs{D_rX_{n+1}(s)}^p]$.
   \begin{align*}
        &\mathbb{E}[\sup_{r\leq s\leq T}\abs{D_r^kX_{n+1}(s)}^p]\\
        =&\mathbb{E}[\sup_{r\leq s\leq T}\abs{\int_{r}^{s}D_r b^k_s(X_n(u))\mathrm{d}u+\sum_{l=1}^{d}(\sigma^{k,l}_r(X_n(r))+\int_{r}^{s}D_r(\sigma^{k,l}_u(X_n(u)))\mathrm{d}W^l_u)}^p]\\
        =&\mathbb{E}[\sup_{r\leq s\leq T}\abs{\int_{r}^{s}\sum_{i=1}^{m}\bar{B}^{n,k,i}_uD_rX^i_n(u)\mathrm{d}u+\sum_{l=1}^{d} ((\sigma^{k,l}_r(X_n(r))+\int_{r}^{s} \sum_{i=1}^{m} \bar{A}_u^{n,k,l,i}D_rX^i_n(u)\mathrm{d}W_u}^p)]\\
        \leq &C_p \mathbb{E}[\sup_{r\leq s\leq T}\left( K^p\abs{\int_{r }^{s }\sum_{i=1}^{m}D_rX^i_n(u)\mathrm{d}u}^p +dK^p\abs{\int_{r }^{s }\sum_{i=1}^{m}D_rX^i_n(u)\mathrm{d}W_u^i}^p+\sum_{l=1}^{d} \sigma^{k,l}_r(X_n(r))\right)]\\
        \leq &C_p K^p\mathbb{E}[\sup_{r\leq s\leq T}\abs{\abs{\int_{r }^{s }\sum_{i=1}^{m}D_rX^i_n(u)\mathrm{d}u}^p}]+dC_pK^p \sum_{i=1}^{m}\mathbb{E}[\abs{\int_{r }^{u }\abs{D_rX^i_n(u)}^2\mathrm{d}u}^{p/2}]+C_p\gamma_p.
   \end{align*}
   其中$\gamma_p=\sup_{n,k}\mathbb{E}[\sup_{0\leq s\leq T}\abs{A^k_s(X_n(s))}^p]<\infty$.
   先估计第一个期望.由Holder不等式可得,
   \begin{align*}
    &\mathbb{E}[\sup_{r\leq s\leq T}\abs{\int_{r }^{s }D_rX^i_n(u)\mathrm{d}u}^p]\\
    \leq & (s-r)^{p-1}\mathbb{E}[\sup_{r\leq s\leq T}\int_{r }^{s }\abs{D_rX^i_n(u)}^p\mathrm{d}u]\\
    \leq &T^{p-1}\mathbb{E}[\int_{r }^{T }\abs{D_rX^i_n(u)}^p\mathrm{d}u].
   \end{align*}\par
   再估计第二个期望.由BDG不等式和Holder不等式可得,
   \begin{align*}
    &\mathbb{E}[\sup_{r\leq s\leq T}\abs{\int_{r }^{s }D_rX^i_n(u)\mathrm{d}W_u^i}^p]\\
    \leq & C_p \mathbb{E}[\left(\int_{r}^{T} \abs{D_rX^i_n(u)}^2 \mathrm{d}u\right)^{p/2}]\\
    \leq & C_p(T-r)^{\frac{p}{2}-1} \mathbb{E}[\int_{r}^{T} \abs{D_rX^i_n(u)}^p \mathrm{d}u].
   \end{align*}
   结合以上可得$\psi_{n+1}(t)\leq c_1+c_2\int_{0}^{t} \psi_n(s)\mathrm{d}s$成立.\par
   由归纳计算可得$\psi_n(t) \leq c_1\sum_{i=1}^{n}\frac{(c_2t)^i}{i!} \leq c_1e^{c_2t}$,也即$X^i_n$的导数在$L^p(\Omega;\mathbb{H})$中是一致有界的,
   结合\ref{lab4}就说明了$X^i(t)\in \mathbb{D}^{1,\infty}$.证明完成.
\end{proof}

\section{例\ref{经典Wiener空间}的扩展}
本节内容主要参考引用自\citep{Shigekawa}和\citep{yanhuang}.\par
在第2节引入的Mailliavin分析框架好处是从性质充分好的光滑随机变量出发,直接给出导数显式表达,相对易于入门,但缺少了一点直观.在\ref{经典Wiener空间}中,可以发现,在经典Wiener空间中,
Malliavin导数算子可以看作是关于样本点的方向导数,这节扩展Wiener空间的定义并从另一角度定义Malliavin导数.\par
首先回顾经典的Wiener空间.令$B=C_0([0,T];\mathbb{R}^d)$,配备一致范数,$\mu$为其上的Wiener测度,当$T<\infty$时,这是一个Banach空间,所以以下都只考虑
有限时间.记$W_t(x)=x_t,x\in B$,则$\{(W_t),t\in [0,T]\}$为典范布朗运动.\par
$B$上的连续线性泛函$\phi$由于连续性自然有可测性,因此是一个随机变量,另一方面,根据Riesz表示定理,$\phi$可以等同于一个Radon测度,可以用
有限Dirac测度的线性组合逼近.注意到对$[0,T]$上的Dirac测度$\delta_{t},t\in [0,T]$,作为一个随机变量服从正态分布,所以可得$\phi$的分布也是正态的,且均值为0.\par
对$\phi\in B^*$,令$\td{\phi}(t)=\phi([0,t])$,令$F(t,x)=\td{\phi}(t)x$应用I\^{o}公式可得,
\[
    \td{\phi}(T)W_T=\int_{0}^{T} W_t \phi(\mathrm{d}t)+\int_{0}^{T}\td{\phi}(t)\mathrm{d}W_t,
\] 
也即
\[
    \int_{0}^{T} W_t \phi(\mathrm{d}t)=\int_{0}^{T}(\td{\phi}(T)-\td{\phi}(t))\mathrm{d}W_t,
\]                          
定义$h_\phi(t)=\int_{0}^{T}(\td{\phi}(T)-\td{\phi}(s))\mathrm{d}s$   ,注意到任意给定一个$x\in B$,$W_{\cdot}(x)=x$,将
$\int_{0}^{T} W_t \phi(\mathrm{d}t)$记为$\phi(W)$,则
$\phi(W)=\int_{0}^{T}\dot{h}_\phi(t)\mathrm{d}W_t$.
定义Cameron-Martin空间为
\[
    H=\{h\in \mathrm{AC}([0,T];\mathbb{R}^d):\dot{h}\in L^2([0,T];\mathbb{R}^d)\},
\]
内积为$\ipr{h}{k}_{H}=\ipr{\dot{h}}{\dot{k}}_{L^2([0,T];\mathbb{R}^d)}$.$(H,\ipr{\cdot}{\cdot}_{H})$是一个Hilbert空间.
记嵌入映射为$\iota:H\hookrightarrow B$,并且其伴随就是上面定义的$\phi\mapsto h_\phi$,记为$\iota^*$.\par
将以上结构抽象出来,就可以给出抽象Wiener空间的定义.
\subsection{抽象Wiener空间}
\begin{definition}
    给定一个Banach空间$B$和一个Hilbert空间$H$,$\mu$为$B$上的测度.若存在连续单射$\iota :H\hookrightarrow B$使得$\iota(H)$在$B$中稠密
    且$\mu$为Gauss测度,即
    \begin{align*}\label{laba.1}
        \int_B \exp(\iu \ipr{x}{y}_H)\mu(\mathrm{d}y)=\exp(-\frac{\nm[H]{y}^2}{2}),y\in B^*\subset H,
\end{align*}
    则称三元组$(B,H,\mu)$为一个抽象Wiener空间.
\end{definition}
这种结构的选择不是偶然的,有多方面原因.一方面,对于有限维的Banach空间总和相同维数的欧氏空间是代数同构,拓扑同胚的,而欧式空间作为一个性质足够好的模型,可以在上面构造Lebsgue测度.Lebsgeue测度最重要的
性质之一是平移不变性:将任意可测集沿任意方向平移任意有限距离,测度保持不变.但对于无限维的Banach空间则不存在平移不变的测度.
退而求其次,希望找到一个测度,在平移下,零测集依然为零测集,称为“拟不变性”,但即便如此,也不一定能做到;只能再退一步,希望找到一个测度,在尽量多的方向上,零测性不变.之所以要求保持零测性,
是因为通常情况下谈论的可测函数$f$实际上在几乎处处意义下的等价类$[f]=\{g:f=g\quad \mu-\mathrm{a.e.}\}$,做微小平移时,零测性的保持可以
使得$\forall g\in [f],g(\cdot+h)\in [f(\cdot +h)]$.具体地说,设存在$\mu-$零测集$A$,$\forall x\in B-A,f(x)=g(x)$,考虑平移,则$\forall x\in B-(A-h),f(x+h)=g(x+h)$,如果$\mu$在平移下保持零测性,则$f(\cdot+h)$和$g(\cdot+h)$依然在$\mu-$a.e.意义下等价,   这样在考虑方向导数时才可以不依赖于等价类代表元的选取.\par

另一方面则是出于对测度的构造的考虑.设$K\subset B^*$是一个有限维的线性子空间,称形如
\[
    C=\{x\in B:(\phi_1(x),\cdots,\phi_n(x))\in E\},n\in \mathbb{N},E\in \mathscr{B}(\mathbb{R}^n),\phi_1,\cdots,\phi_n\in K,
\]
的集合为以$K$为底的柱集,$\sigma(K)$为以$K$为底的集合生成的$\sigma$代数.在其上可以定义测度,称为柱测度.但需要注意的是,一般$\sigma(K)$比$\mathscr{B}(B)$小得多,
实际上$\mathscr{B}(B)=\sigma(\cup_{K} \sigma(K))$.如何判断一个柱测度能否延拓到整个$\mathscr{B}(B)$上?Gross定理给出了充分条件,简单地说即$B$是某个Hilbert空间
关于某范数的完备化.具体细节见\citep{yanhuang}的 第一章$\S$4节和\citep{stroock}的第三章.\par
在此先简要介绍无穷维空间上的测度涉及的一些概念.
\begin{definition}
    给定一个Hilbert空间和其上的Borel概率测度$\mu$.\par
    称$m\in H$是$\mu$的均值向量,是指:对任意$x\in H$,函数$z\mapsto \langle x,z\rangle$是可积的,且
    \[
        \ipr{m}{x}=\int_H \ipr{x}{z}\mu(\mathrm{d}z).
    \]
    进一步地,称对称正定线性算子$B$为$\mu$的协方差算子,是指:对任意的$x,y\in H$
    \[
        \ipr{Bx}{y}=\int_H \ipr{z-m}{x}\ipr{z-m}{y}\mu(\mathrm{d}z).
    \]
\end{definition}
\begin{definition}
    给定一个Hilbert空间.称Borel概率测度$\mu$是$H$上的Gauss测度,是指:任意$x\in H=H^*$作为一个随机变量服从正态分布.
\end{definition}

\begin{remark}
    均值向量和协方差算子不一定存在.但如果$\mu$有一阶矩,即$\int_H \nm[H]{x}\mu(\mathrm{d}x)<\infty$,则均值向量存在;如果$\mu$有二阶矩,
    即$\int_H \nm[H]{x}^2\mu(\mathrm{d}x)<\infty$,则协方差算子存在.$\mu$为$H$上Gauss测度的充要条件是存在$m \in H$和对称正定迹算子$B$使得其Fourier变换
    \[
        \hat{\mu}(x)\coloneq \int_H e^{\iu \ipr{x}{z}}\mu(\mathrm{d}z)=e^{\iu \ipr{m}{x}-\frac{1}{2}\ipr{Bx}{x}}.
    \]
\end{remark}
\begin{definition}
    给定可分Banach空间$B$.称Borel概率测度$\mu$是$B$上的对称Gauss测度,是指:任意$\phi\in B^*$作为一个随机变量都服从零均值的正态分布.
\end{definition}

回到抽象Wiener空间.设$(B,H,\mu)$是一个抽象Wiener空间,对任意$\phi\in B^*$,
\[
    \int_B \phi(x)\psi(x)\mu(\mathrm{d}x)=\ipr{\iota^* \phi}{\iota^* \psi}_{H^*},
\]
则$\iota^*\phi\mapsto \phi$g给出了$\iota^* (B^*)$到$L^2(B,\mathscr{B},\mu)$的线性等距同构,由稠密性可延拓为$H^*$到$L^2(B,\mathscr{B},\mu)$的线性等距同构,记该同构为$I_1$.

以下定理再次说明了抽象Wiener空间定义的合理性和必要性.
\begin{theorem}
    设$(B,H,\mu)$是一个抽象Wiener空间.对任意$h\in H$,平移测度$\mu(\cdot-h)$与$\mu$相互绝对连续.且其Radon-Nikod\'{y}m导数为
    \[
        \frac{\mathrm{d}\mu(\cdot-h)}{\mathrm{d}\mu}(x)=e^{-\frac{\nm[H]{h}^2}{2}+I_1(h)(x)},
    \]
\end{theorem}
\begin{proof}
    首先注意到$\forall h\in H^*$,可以用一列$B^*$中的元素去逼近,所以$I_1(h)$也是正态的,并且对任意的$\phi\in B^*$,$(\phi,I_1(h)$是二元正态的,
    显然均值为0,而协方差为$\mathbb{E}[\phi I_1(h)]=\ipr{\iota^*\phi}{h}_{H^*}=\ipr{\phi}{\iota h}=\phi(h)$.
    所以其特征函数为
    \[
        \int_B e^{\iu(\xi_1 \phi(x)+\xi_2 I_1(h)(x))}\mu(\mathrm{d}x)=e^{-\frac{1}{2}(\nm[H^*]{\phi}^2\xi_1^2+2\phi(h)\xi_1\xi_2+\nm[H]{h}^2\xi_2^2)},
    \]
    将其解析延拓到$\mathbb{C^2}$上并取$\xi_1=1,\xi_2=-\iu$,
\begin{align*}
    &\int_B e^{\iu\phi(x) + I_1({h})(x)}\mu(\mathrm{d}x) \\
    =& e^{-\frac{1}{2}|\nm[H^*]{\phi}^2 + \iu\phi(x) + \frac{1}{2}\nm[H]{h}^2} \\
    =& e^{\frac{1}{2}\nm[H]{h}^2}\int_B e^{\iu\phi(x) + \iu\phi(h)}\mu(\mathrm{d} x).
\end{align*}
这就证明了结论.
\end{proof}
\begin{remark}
    以上定理说明了Gauss测度沿$H$中的方向平移具有拟不变性,实际上,在任意其他方向平移得到的测度一定与$\mu$是互相奇异的,具体参见\citep{stroock}的定理3.3.5.
\end{remark}

下面的定理说明了一般Banach空间上的Gauss测度和有限维空间上的Gauss测度基本具有相同的可积性.
\begin{theorem}[Fernique定理]
    设$\mu$为Banach空间上$B$的Gauss测度,则存在常数$\lambda$,使得
    \[
        \int_B e^{\lambda \nm[B]{x}^2}\mu(\mathrm{d}x)<\infty.
    \]
\end{theorem}
\begin{remark}
    实际上对$B$上任一半范数都有类似的可积性,只是常数$\lambda$可能不同.
\end{remark}

\subsection{“平移”算子和导数算子}
下面从另一个角度定义Malliavin导数,首先和第2节一样,整个框架的基础是Gauss概率空间$(\Omega,\mathscr{F},\mathbb{P};\mathbb{H})$,其中$(\Omega,\mathscr{F},\mathbb{P})$是一个完备
概率空间,$\mathbb{H}$是一个可分Hilbert空间,$W=\{W(h):h\in \mathbb{H}\}$是一个等距Gauss过程.抽象Wiener空间$(B,H,\mu)$也是一个Gauss概率空间,等距Gauss过程由$W(\phi)=\phi,\phi\in B^*\subset H$延拓得到,若$\mathscr{F}$是$\sigma(W)$的完备化,则称
该Gauss概率空间不可约.\par
基本的想法是类似于G\^{a}teaux方向导数那样定义关于样本点的方向导数,但对于一般的概率测度空间而言,并没有天然的代数结构,也就不能进行加法或者说平移,因此需要考虑一个合适的样板模型,在这个模型上做加法运算,然后建立和原来概率空间和随机变量的对应,
注意到任意可分的Hilbert空间均同构于$l^2$,所以一个合适的Gauss概率空间即$(\mathbb{R}^{\mathbb{N}},\mathscr{B}(\mathbb{R})^{\mathbb{N}},\gamma^{\mathbb{N}};l^2)$.其中$\gamma$为一维Gauss测度,
$(\mathbb{R}^{\mathbb{N}},\mathscr{B}(\mathbb{R})^{\mathbb{N}},\gamma^{\mathbb{N}})$为$(\mathbb{R},\mathscr{B}(\mathbb{R}),\gamma)$的无穷独立乘积空间.\par
给定$\mathbb{H}$的一个规范正交基$\{e_i\}$,定义映射$T$为
\begin{align*}
    T:\Omega&\to \mathbb{R}^{\mathbb{N}}\\
    \omega&\mapsto (W(e_i)(\omega)).
\end{align*}
这是一个保测映射,即$\gamma^{\mathbb{N}}=\mathbb{P}\circ T^{-1}$.另外,任意$1\leq p\leq \infty$,$\phi\in L^p((\mathbb{R}^{\mathbb{N}},\mathscr{B}(\mathbb{R})^{\mathbb{N}},\gamma^{\mathbb{N}})$,定义 
$T_*\phi=\phi\circ T$,则$T_*$是$L^p((\mathbb{R}^{\mathbb{N}},\mathscr{B}(\mathbb{R})^{\mathbb{N}},\gamma^{\mathbb{N}})$到$L^p(\Omega,\mathscr{F},\mathbb{P})$的同构.\par
下面将$\mathbb{R}^{\mathbb{N}}$上的加法结构“移植”到$\Omega$上,更准确地说是找一个合适的映射来代替函数的平移$f\mapsto f(\cdot+h)$.首先定义
\[
    L^{\infty -}(\Omega,\mathscr{F},\mathbb{P})=\cap_{1<p<\infty}L^p(\Omega,\mathscr{F},\mathbb{P})
\]
和
\[
    L^{1+}(\Omega,\mathscr{F},\mathbb{P})=\cup_{1<p<\infty}L^p(\Omega,\mathscr{F},\mathbb{P}),
\]
分别称为投影极限和归纳极限,$L^{\infty -}(\Omega,\mathscr{F},\mathbb{P})$关于有限乘积封闭,且乘法运算连续.另外定义$\mathbb{R}^{\mathbb{N}}$上泛函的平移:$\tau_h:f\mapsto f(\cdot+h)$.
\begin{definition}
    给定不可约Gauss概率空间$(\Omega,\mathscr{F},\mathbb{P};\mathbb{H})$,取定$\mathbb{H}$的一个规范正交基$\{e_i\}$,$\mathbb{H}$到$l^2$的同构
    记为$J:\mathbb{H}\to l^2$.对任意$h\in \mathbb{H}$,定义$L^{1+}(\Omega,\mathscr{F},\mathbb{P})$上的算子
    \[
        \rho_h=T_*\circ \tau_{J(h)}\circ T_*^{-1}
    \]
    称$\rho$为$\mathbb{H}$中加群的典则表示.
\end{definition}
\[\begin{tikzcd}
	{L^p(\mathbb{R}^{\mathbb{N}})} && {L^p(\mathbb{R}^{\mathbb{N}})} \\
	\\
	{L^p(\Omega)} && {L^p(\Omega)}
	\arrow["{\tau_{J(h)}}", from=1-1, to=1-3]
	\arrow["{T_*}", from=1-3, to=3-3]
	\arrow["{T^{-1}_*}", from=3-1, to=1-1]
	\arrow["{\rho_h}", from=3-1, to=3-3]
\end{tikzcd}\]
\begin{remark}
    为更清晰,这里给出逐点的表示.设$F\in L^p(\Omega,\mathscr{F},\mathbb{P}),1<p<\infty$.则
    \begin{align*}
        (\rho_hF)(x)&=(T_*\circ \tau_{J(h)}\circ T_*^{-1} f)(\omega)\\
        &=T_*((T^{-1}_* F)(\cdot+J(h)))(\omega)\\
        &=((T^{-1}_* F)(\cdot+J(h))(T\omega)\\
        &=(T^{-1}_* F)(T\omega+J(h)).
    \end{align*}\par
    $\{\rho_h \}$也有群结构,即$\rho_{h+g}=\rho(h)\rho_g,h,g\in \mathbb{H}$.
\end{remark}
这种取定一个性质足够好的样板模型,利用模型的结构和性质来定义一些对象再拉回到原空间的做法不是偶然或独立的,例如定义流形的光滑结构和光滑映射时也是将其归结为欧式空间
之间的光滑映射.

下面的定理说明测度$\mathbb{P}$关于这种平移的替代$\rho_h$具有拟不变性.
\begin{theorem}[Cameron-Martin定理]
    设$(\Omega,\mathscr{F},\mathbb{P};\mathbb{H})$为不可约Gauss概率空间,$\rho$为$\mathbb{H}$中加群的典则表示,定义指数泛函
    \[
        \mathcal{E}(h)=e^{W(h)-\frac{1}{2}\nm[\mathbb{H}]{h}^2},h\in \mathbb{H}.
    \]
    则$\mathcal(E)(h)\in L^{\infty-}(\Omega)$,且 
    \[
       \nm[p]{\mathcal{E}(h)}\leq e^{\frac{p-1}{2}\nm[\mathbb{H}]{h}^2},1<p<\infty.
    \]
    对任意$F\in L^{1+}(\Omega)$,
    \[
        \int_\Omega \rho_h F \mathrm{d}\mathbb{P}=\int_\Omega F \mathcal{E}(h)\mathrm{d}\mathbb{P}.
    \]
    另外$\lim_{t\downarrow 0}\frac{\mathcal{E}(th)-1}{t}=W(h)$.
\end{theorem}
证明只需考虑有限维欧氏空间情形,取极限得到无穷独立乘积中的结论,再利用同构回到一般的不可约Gauss概率空间.\par

有了以上准备,可以对不可约Gauss概率空间上的随机变量定义Malliavin导数.
\begin{definition}
    设$(\Omega,\mathscr{F},\mathbb{P};\mathbb{H})$为不可约Gauss概率空间,$\rho$为$\mathbb{H}$中加群的典则表示.
    $F\in \mathrm{Srv}$为光滑随机变量,定义其导数为$DF\in \mathrm{Srv}(\mathbb{H})$,并由下式唯一确定:
    \[
        \ipr{DF}{h}_{\mathbb{H}}=\lim_{\epsilon\downarrow 0}\frac{\rho_{\epsilon h}F-F}{\epsilon}.
    \]
\end{definition}
可以验证,对于光滑随机变量,该定义与第二节中给出的定义是完全一致的.这样,导数算子的闭延拓,散度算子的定义等都可以往下继续构建,不再赘述.\par

再次回顾抽象Wiener空间和其上的泛函$\phi$,第二章已经有了G\^{a}teaux方向导数的概念,但G\^{a}teaux方向导数实际是属于$B^*$的,所以Malliavin导数比G\^{a}teaux导数
更强,因为$B^*\subset H^*=H$.


                                                                                                                                                                                                                                                                                                                                                                                                                                                             














