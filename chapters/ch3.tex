\chapter{外在导数,内蕴导数和线性泛函导数}
本章内容主要参考引用自\citep{wrde},\citep{mpsa},\citep{brw}和\citep{ddsde}.\par
Lions导数可以看作是Frechet导数的延伸,这一章则主要介绍Gateaux导数或方向导数的概念在测度变量的函数
上的应用.
参考文献是在Riemann流形上对一般情况定义了几种导数,为简单起见,此处只考虑欧氏空间$\mathbb{R}^d$的情形.$\mathscr{M}$表示$\mathbb{R}^d$上有限测度全体,
记$\mathscr{M}_p=\{\mu\in\mathscr{M}:\mu(\abs{\cdot}^p)<\infty\}$表示$\mathbb{R}^d$上$p$阶矩有限测度的全体,$p\in [0,\infty)$,配备通常的加法,减法和数乘.对$p>0$,定义Wasserstein距离
\[
    W_p(\mu,\nu)=(\mu(\mathbb{R}^d)\wedge \nu(\mathbb{R}^d))W_p(\frac{\mu}{\mu(\mathbb{R}^d)},\frac{\nu}{\nu(\mathbb{R}^d)})+\abs{\mu(\mathbb{R}^d)-\nu(\mathbb{R}^d)},
\]
当$p=0$时,定义Prokhorov度量
\[
    W_0(\mu,\nu )=\inf \{\epsilon>0:\mu(A)\leq\epsilon+\nu(A^\epsilon),\nu(A)\leq\epsilon+\mu(A^\epsilon),A\in \mathscr{B}(\mathbb{R}^d)\},
\]
其中$A^\epsilon=\{x\in \mathbb{R}^d:d(x,A)<\epsilon\}$.\par
对于任意$p\in [0,\infty)$,$(\mathscr{M}_p,W_p)$是一个可分完备度量空间.
\section{定义}
我们首先给出外在导数的定义.
\begin{definition}
    设$p\geq 0$.称$f:\mathscr{M}_p\to \mathbb{R}$在$\mathscr{M}_p$上是外在可导的,是指:任意$(x,\eta)\in \mathbb{R}^d\times \mathscr{M}_p$极限
    \[
        D^Ef(\eta)(x)\coloneq \lim_{\epsilon\downarrow 0} \frac{f(\eta+\epsilon \delta_x)-f(\eta)  }{\epsilon}
    \]
    存在,其中$\delta_x$表示$x\in \mathbb{R}^d$处的Diarc质量.$D^E f$称为$f$的外在导数.
\end{definition}
\begin{remark}
    在内蕴导数的定义的基础上,定义若干常用的函数类.\par
    称函数$f:\mathscr{M}_p\to \mathbb{R}\in C^{E,1}(\mathscr{M}_p)$,是指:$f$在$\mathscr{M}_p$上是外在可导的,
    且映射$(x,\eta)\mapsto D^E f(\eta)(x)$是$\mathbb{R}^d\times \mathscr{M}_p$上的连续函数.\par
    称函数$f:\mathscr{M}_p\to \mathbb{R}\in C^{E,1}_K(\mathscr{M}_p)$,是指:$f\in C^{E,1}(\mathscr{M}_p)$,且对任意紧集
    $K\subset \mathscr{M}_p$,存在常数$C$,使得任意 $\eta\in K$,$\abs{D^Ef(\eta)(x)}\leq C(1+\abs{x}^p)$.\par
    称函数$f:\mathscr{M}_p\to \mathbb{R}\in C^{E,1,1}(\mathscr{M}_p)$,是指:$f\in C^{E,1}(\mathscr{M}_p)$,$D^E f(\eta)(x)$关于$x$可微,
    且$\nabla D^E f(\eta)(x)$是$\mathbb{R}^d\times \mathscr{M}_p$上的连续函数.\par
    此外,$C^{E,1}_b(\mathscr{M}_p),C^{E,1,1}_b(\mathscr{M}_p)$分别表示$C^{E,1}(\mathscr{M}_p),C^{E,1,1}(\mathscr{M}_p)$和$\mathscr{M}_p$上有界函数的交集.
\end{remark}

容易看出,外在导数的定义不能直接限制在$p$阶矩有限的的概率测度组成的集合$\mathscr{P}_p\subset \mathscr{M}_p$上,
因此对以上定义稍作修改,引入凸外在导数的概念.
\begin{definition}
    设$p\geq 0$.称函数$f:\mathscr{P}_p\to \mathbb{R}$凸外在可导,是指:对任意$(x,\mu)\in \mathbb{R}^d\times \mathscr{P}_p$,极限
    \[
        \tilde{D}^E f(\mu)(x)\coloneq \lim_{s\downarrow 0}\frac{f((1-s)\mu+s\delta_x)}{s}
    \]
    存在,$\tilde{D}^E f$称为$f$的凸外在导数.
\end{definition}

\begin{remark}
    类比外在导数,同样可以类似的函数空间.\par
    称函数$f:\mathscr{P}_p\to \mathbb{R}\in C^{E,1}(\mathscr{P}_p)$,是指:$f$在$\mathscr{P}_p$上是外在可导的,
    且映射$(x,\eta)\mapsto \tilde{D}^E f(\eta)(x)$是$\mathbb{R}^d\times \mathscr{P}_p$上的连续函数.\par
    称函数$f:\mathscr{P}_p\to \mathbb{R}\in C^{E,1}_K(\mathscr{P}_p)$,是指:$f\in C^{E,1}(\mathscr{P}_p)$,且对任意紧集
    $K\subset \mathscr{P}_p$,存在常数$C$,使得任意 $\eta\in K$,$\abs{\tilde{D}^Ef(\eta)(x)}\leq C(1+\abs{x}^p)$.\par
    称函数$f:\mathscr{P}_p\to \mathbb{R}\in C^{E,1,1}(\mathscr{P}_p)$,是指:$f\in C^{E,1}(\mathscr{P}_p)$,$\tilde{D}^E f(\eta)(x)$关于$x$可微,
    且$\nabla \tilde{D}^E f(\eta)(x)$是$\mathbb{R}^d\times \mathscr{P}_p$上的连续函数.\par
    此外,$C^{E,1}_b(\mathscr{P}_p),C^{E,1,1}_b(\mathscr{P}_p)$分别表示$C^{E,1}(\mathscr{P}_p),C^{E,1,1}(\mathscr{P}_p)$和$\mathscr{P}_p$上有界函数的交集.
\end{remark}
虽然凸外在导数和外在导数定义上稍有差别,但存在着紧密关联:若函数$f:\mathscr{M}_p\to \mathbb{R}\in C^{E,1}_b(\mathscr{M}_p)$,则$f|_{\mathscr{P}_p}\in C^{E,1}(\mathscr{P}_p)$,且对任意
$\mu\in \mathscr{P}_p$,
\[
    \tilde{D}^E f|_{\mathscr{P}_p}(\mu)=D^E f(\mu)-\mu(D^E f(\mu)),
\]
在这个意义下,可以将凸外在导数看作外在导数的中心化.

给定向量场$\phi:\mathbb{R}^d\to \mathbb{R}^d\in \mathscr{B}(\mathbb{R}^d\to \mathbb{R}^d)$,定义
\[
    \phi_s(x)\coloneq x+s\phi(x),s\geq 0,
\]
也即$\phi_s=\mathrm{id}_{\mathbb{R}^d}+s\phi,s\geq 0$.
\par
对于任意测度$\eta\in \mathscr{M}_p,p\in (0,2]$,定义$\eta$处的切空间为$L^2(\mathscr{B}(\mathbb{R}^d\to \mathbb{R}^d);\eta)$,
这是一个Hilbert空间,可以应用Riesz表示定理,将其上的连续线性泛函视为其中一个元素.
\begin{definition}
    设$p\in [0,2]$.称函数$f:\mathscr{M}_p\to \mathbb{R}$在$\eta\in \mathscr{M}_p$内蕴可导,是指:对于任意
    $\phi\in L^2(\mathscr{B}(\mathbb{R}^d\to \mathbb{R}^d);\eta)$,极限
    \[
        D^I_\phi f(\eta)\coloneq \lim_{s\downarrow 0}\frac{f(\eta\circ \phi_s^{-1})-f(\eta)}{s}
    \]
    存在,且是$\phi$的连续线性泛函.此时存在唯一的$D^I f(\eta)\in L^2(\mathscr{B}(\mathbb{R}^d\to \mathbb{R}^d);\eta)$,使得
    \[
        D^I_\phi f(\eta)=\ipr{D^I f(\eta)}{\phi}_{L^2(\mathscr{B}(\mathbb{R}^d\to \mathbb{R}^d);\eta)}=\int_{\mathbb{R}^d} \ipr{D^I f(\eta)}{\phi} \mathrm{d}\eta,
    \]
    称$D^I f(\eta)$为$f$在$\eta$处的内蕴导数,若对任意$\mu\in \mathscr{M}_p$,$f$在$\mu$处都是内蕴可导的,
    则称$f$在$\mathscr{M}_p$上内蕴可导或简称内蕴可导.
\end{definition}
\begin{remark}
    由于内蕴导数的定义不依赖于$\mathscr{M}_p$的线性结构,因此对$\mathscr{P}_p$上的实值函数有完全相同的定义.
\end{remark}

作为比较,这里给出Lions导数的另一种定义.在第二章中,是将测度变量函数$f:\mathscr{P}_2\to \mathbb{R}$提升为以随机变量
为变量的函数$\td{f}=f\circ \mathscr{L}:L^2(\Omega;\mathbb{R}^d)\to \mathbb{R}$,然后用随机变量空间上的Frechet导数来刻画$f$的局部性态.
在考虑差$\td{f}(X+Y)-\td{f}(X)$以及$\nm[L^2]{Y}$逐渐趋于$0$时,如果固定一个方向取$Y=s\phi(X)$,则
\begin{equation}
    \begin{aligned}
        \td{f}(X+Y)-\td{f}(X)&=\td{f}(X+s\Phi(X))-\td{f}(X)\\
        &=f(\mathscr{L}(X)\circ (\mathrm{id}+s\phi)^{-1})-f(\mathscr{L}(X))\\
        &=f(\mathscr{L}(X)\circ \phi_s^{-1})-f(\mathscr{L}(X)),
    \end{aligned}\notag
\end{equation}
这就是内蕴导数的形式.而Lions导数本质是Frechet导数,应该考虑任意方向,所以在令$Y$的二阶矩趋于$0$的过程不应该通过
给一个固定的函数做伸缩来实现,而应该让函数本身充分"小",用严格的数学语言来叙述即如下定义.
\begin{definition}\label{L可微的第二定义}
    给定$p\in [0,2]$.称函数$f:\mathscr{M}_p\to \mathbb{R}$在$\eta\in \mathscr{M}_p$L-可导,是指:$f$是内蕴可导的,且
    \[
        \lim_{\nm[L^p(\eta)]{\phi}\downarrow 0}\frac{\abs{f(\eta\circ (\mathrm{id}+\phi)^{-1})-f(\eta)-D^I_\phi f(\eta)}}{\nm[L^p(\eta)]{\phi}}=0,
    \]
    若$f$在任意$\eta\in \mathscr{M}_p$都L-可导,则称$f$是L-可导的,此时将$D^I f$记为$D^L f$.
\end{definition}
\begin{remark}
    根据第二章的相关讨论,在这个定义中,要求$f$内蕴可导是自然的;另外,在这个L-可导的定义下,L-可导显然强于
    内蕴可导,而当$f$L-可导时,其L-导数就是内蕴导数. 以上定义对$\mathscr{M}_p$上的函数同样成立.\par
    称$f\in C^{L,1}(\mathscr{M}_p)$,是指:$f$是L-可导的,且存在$\eta$版本的$D^L f(\eta)(\cdot)$使得$(x,\eta)\mapsto D^L f(\eta)(x)$是$\mathbb{R}^d\times \mathscr{M}_p$上的连续映射;若在此基础上,$D^L f$还是有界的,
    则记$f\in C^{L,1}_b(\mathscr{M}_p)$.\par
    以上关于L-导数的定义和函数类均可完全限制在$\mathscr{P}_p$的情形下给出对应的定义和记号.
\end{remark}
    

在第一章中,对于Banach空间之间的映射如果是可微的并满足合适的条件,则有中值定理成立.对于$\mathscr{M}_p$或$\mathscr{P}_p$,则可以
用"中值定理"来定义一点处的导数,从而一定程度上在形式上与Banach空间上的Frechet导数保持一致.
\begin{definition}
    设$p\in [0,\infty)$.给定$f:\mathscr{M}_p\to \mathbb{R}$,称可测函数
    $D^F f(\eta):\mathbb{R}^d\to \mathbb{R}$是$f$在$\eta$处的线性泛函导数,是指:对任意$L>0$,存在$C_L$使得
    \[
        \sup_{\eta(\abs{\cdot}^p)\leq L} \abs{D^F f(\eta)(y)}\leq C_L(1+\abs{y}^p),y\in \mathbb{R}^d,
    \]
    且对任意$\mu,\nu\in \mathscr{M}_p$,
    \[
        f(\mu)-f(\nu)=\int_{0}^{1}\int_{\mathbb{R}^d} D^F f(\nu+t(\mu-\nu))(y)(\mu-\nu)(\mathrm{d}y) \mathrm{d}t.
    \]
\end{definition}
\begin{remark}
    由于只涉及到凸组合,所以线性泛函导数的定义对$\mathscr{P}_p$上的实值函数完全成立.
\end{remark}

以上定义的几种导数并不是完全孤立的,在合适的条件下,它们之间可以相互表达.将外在导数和凸外在导数看作一类,内蕴导数和L-导数
看作一类,线性泛函导数看作一类,这三类导数之间都存在一定的联系.\par

由于L-导数可以视为更强的内蕴导数,所以这里只讨论外在导数和L-导数的关系.从定义的形式上来看,外在导数的定义要比
L-导数的计算"简单"一些,所以希望能找到它们间的关联,从而通过计算外在导数来简化L-导数的计算.事实上,有以下定理.
\begin{theorem}
    设$p\in[0,2]$.\ 若$f\in C^{E,1,1}_b(\mathscr{M}_p)$,则$f\in C^{L,1}_b(\mathscr{M}_p)$,
    且
    \[
        D^L f(\eta)=\nabla [D^E f(\eta)],\eta\in \mathscr{M}_p
    \]
\end{theorem}
证明思路和细节参见\citep{mpsa}的定理2.1和\citep{wrde}的定理2.1(3).\par
实际上,上述定理还说明在一定条件下,测度变量的函数在一点处的L-导数可以视为某标量函数的梯度;反过来,如果
已知L-导数,则可以通过积分或中值定理计算,在相差一个常数的意义下得到外在导数,即
\[
    D^E f(\eta)(x)=D^E f(\eta)(0)+\int_{0}^{1} D^L f(\eta )(x)\cdot x \mathrm{d}x.
\]

而对于线性泛函导数,在多数情况下,如果只根据定义只能验证而非计算一个测度变量函数的线性泛函导数,但以下定理说明
对一大类函数,外在导数若存在,则外在导数就是线性泛函导数.
\begin{theorem}
    若函数$f\in C^{E,1}_K(\mathscr{M}_p)$,则$f$的线性泛函导数存在,且$D^F f=D^E f$.
\end{theorem}
\section{链式法则}
由于在一定条件下,三类导数可以相互表示,所以这里只给出L-导数的链式法则,也是推论\ref{cr1}的一般情形.
\begin{theorem}
    设$p\geq 1$.$f:\mathscr{P}_p\to \mathbb{R}$是连续函数,$(X_s)_{s\in[0,1]}$是一族$\mathbb{R}^d$值的随机变量,$X_s$在$L^p$意义下
    关于$s$连续,
    $\dot{X}_0\coloneq \lim_{s\downarrow 0}\frac{X_s-X_0}{s}$在$L^p(\Omega;\mathbb{R}^d)$中存在.下列两个条件至少满足其一:\par
    (1)\ $\mu_0=\mathscr{L}(X_0)$无原子,$f$是L-可微的且$D^L f(\mu_0)$存在一个连续版本以及常数$C$使得
    \[
        \abs{D^L f(\mu_0)(x)}\leq C(1+\abs{x}^{p-1}),
    \]\par
    (2)\ $f$在$\mu_0$的一个邻域$O$内是L-可微的,$D^L f$有一个版本使得$\mathbb{R}^d\times O\ni (x,\mu)\mapsto D^L f(\mu)(x)$连续,且存在常数 $C$使得
    \[
        \abs{D^L f(\mu_0)(x)}\leq C(1+\abs{x}^{p-1}).
    \] \par
    则
    \begin{align}\label{cr2}
        \lim_{s\downarrow 0}\frac{f(\mathscr{L}(X_s))-f(\mu_0)}{s}=\mathbb{E}[\ipr{D^L f(\mu_0)(X_0)}{\dot{X}_0}].
    \end{align}
\end{theorem}

\begin{proof}\ \par
    (1)\ 当$\mu_0$无原子时,完全类比于第二章引理\ref{lm3.1}的讨论可得,对于任意$n\in \mathbb{N}$,存在可测映射$S_n:\Omega\to \mathbb{R}^d$和$T_n:\mathbb{R}^d\to \Omega$,使得
    \begin{align}
        &\mathbb{P}(T_n\circ S_n =\mathrm{id}_\Omega)=\mu_0(S_n\circ T_n=\mathrm{id}_{\mathbb{R}^d})=1,\\
        &\mathbb{P}=\mu_0\circ T_n^{-1}, \mu_0=\mathbb{P}\circ S_n^{-1},\label{4.2}\\
        &\nm[L^\infty(\mathbb{P})]{X_0-S_n}+\nm[L^\infty(\mu_0)]{\mathrm{id}_{\mathbb{R}^d}-X_0\circ T_n}\leq\frac{1}{n}.\label{4.3}
    \end{align}
    由L-可导的定义\ref{L可微的第二定义}可知,存在一个非增的函数$h$满足当$s\to 0$,$h(s)\to 0$,使得
    \[
        \sup_{\nm[L^p(\mu_0)]{\phi}\leq r} \abs{f(\mu_0\circ(\mathrm{id}_{\mathbb{R}^d}+\phi)^{-1})-f(\mu_0)-D^L_\phi f(\mu_0)}\leq rh(r),
    \]
    由$\mathscr{L}(X_s-X_0)\in \mathscr{P}_p$和式\ref{4.2}可得
    $\phi_{n,s}\coloneq (X_s-X_0)\circ T_n\in L^p(\mu_0)$,这是因为$\nm[L^p(\mu_0)]{\phi_{n,s}}=\nm[L^p(\mathbb{P})]{X_s-X_0}$,
    对任意的$A\in \mathscr{B}(\mathbb{R}^d)$
    \begin{equation}
        \begin{aligned}
            \mathscr{L}(S_n+X_s-X_0)(A)&=\mathbb{P}((S_n+X_s-X_0)^{-1}(A))\\
            &=(\mu_0\circ T_n^{-1})((S_n+X_s-X_0)^{-1}(A))\\
            &=\mu_0(T_n^{-1}((S_n+X_s-X_0)^{-1}(A)))\\
            &=\mu_0(((S_n+X_s-X_0)\circ T_n)^{-1(A)})\\
            &=\mu_0((\mathrm{id}_{\mathbb{R}^d}+(X_s-X_0)\circ T_n)^{-1}(A))\\
            &=\mu_0((\mathrm{id}_{\mathbb{R}^d}+\phi_{n,s})^{-1}(A))\\
            &=(\mu_0\circ (\mathrm{id}_{\mathbb{R}^d}+\phi_{n,s})^{-1})(A).
        \end{aligned}
    \end{equation}
    由条件可知,存在$\delta\in[0,1]$使得任意$s\in(0,\delta)$,
    \[
            \nm[L^p(\mathbb{P})]{\frac{X_s-X_0}{s}-\dot{X}_0}\leq \nm[L^2(\mathbb{P})]{\dot{X_0}}\implies \nm[L^2(\mathbb{P})]{X_s-X_0}\leq s(2\nm[L^2(\mathbb{P})]{\dot{X}_0}\vee 1)=cs,
    \]
    另一方面,
    \begin{equation}
        \begin{aligned}
            D^L_{\phi_{n,s}}f(\mu_0)&=\ipr{D^L f(\mu_0)}{\phi_{n,s}}_{L^2(\mu_0)}\\
            &=\int_{\mathbb{R}^d} \ipr{D^L f(\mu_0)(x)}{\phi_{n,s}(x)}\mathrm{d}\mu_0(x)\\
            &=\int_{\mathbb{R}^d} \ipr{D^L f(\mu_0)(x)}{\phi_{n,s}(x)}\mathrm{d}(\mathbb{P}\circ S_n^{-1})(x)\\
            &=\mathbb{E}[\ipr{D^L f(\mu_0)(S_n)}{\phi_{n,s}(S_n)}]\\
            &=\mathbb{E}[\ipr{D^L f(\mu_0)(S_n)}{X_s-X_0}].
        \end{aligned}
    \end{equation}
    综上可得,
    \begin{equation}
        \begin{aligned}
            &\abs{f(\mathscr{L}(S_n+X_s-X_0))-f(\mathscr{L}(X_0))-\mathbb{E}[\ipr{D^L f(\mu_0)(S_n)}{X_s-X_0}]}\\
            =&\abs{f(\mu_0\circ (\mathrm{id}_{\mathbb{R}^d}+\phi_{n,s})^{-1})-f(\mu_0)-D^L_{\phi_{n,s}} f(\mu_0)}\\
            \leq &\nm[L^p(\mu_0)]{\phi_{n,s}}h(\nm[L^p(\mu_0)]{\phi_{n,s}})\\
            =&\nm[L^p(\mathbb{P})]{X_s-X_0}h(\nm[L^p(\mathbb{P})]{X_s-X_0}),s\in[0,\frac{1}{c}]
        \end{aligned}
    \end{equation}
    由式\ref{4.3}可知,当$n\to\infty$,$S_n$几乎处处收敛到$X_0$,且由控制收敛定理加上$f$及其导数的连续性可得
    \[
        \abs{f(\mathscr{L}(X_s)-f(\mu_0)-\mathbb{E}[\ipr{D^L f(\mu_0)(X_0)}{X_s-X_0}]}\leq \nm[L^p(\mathbb{P})]{X_s-X_0}h(\nm[L^p(\mathbb{P})]{X_s-X_0}),
    \]
    两边同时除以$s$再令$s\to 0$即可. \par
    (2)\ 若$\mu_0$不是无原子的,取一个独立于$(X_s)$且分布无原子的随机变量$X$,则随机变量$X_0+\epsilon X+r(X_s-X_0)$的分布总是无原子的.当$\delta$充分小时,
    应用中值定理可得,
    \begin{equation}
        \begin{aligned}
            &f(\mathscr{L}(X_s+\epsilon X))-f(\mathscr{L}(X_0+sX))\\
            =&\int_{0}^{1}\frac{\mathrm{d}}{\mathrm{d}\delta    }|_{\delta=0}f(\mathscr{L}(X_0+\epsilon X+(r+\delta)(X_s-X_0)))\mathrm{d}r\\
            =&\int_{0}^{1}\mathbb{E}[\langle D^L f(\mathscr{L}(X_0+\epsilon X+(r+\delta)(X_s-X_0)))(X_0+\epsilon X+(r+\delta)(X_s-X_0)),X_s-X_0\rangle],
        \end{aligned}
    \end{equation}
    令$\epsilon\to 0$即可.
\end{proof}

\section{线性泛函导数和内蕴导数的补充}

本节内容取自\citep{pfjp}.本节主要是补充对线性泛函导数和内蕴导数的其他观点,空间取为$d$维环面$\mathbb{T}^d$,$\mathscr{P}(\mathbb{T}^d)$表示$\mathbb{T}^d$上全体概率测度.
\begin{definition}
    称函数$U:\mathscr{P}(\mathbb{T}^d)\to \mathbb{R}$是$C^1$的,是指:存在连续映射$\frac{\delta U}{\delta m}:\mathscr{P}(\mathbb{T}^d)\times \mathbb{T}^d\to \mathbb{R}$,使得对任意$m,m'$都有
    \[
        \int_{\mathbb{T}^d}\frac{\delta U}{\delta m}(m,y)\mathrm{d}m(y)=0
    \]
    且
    \[
        \lim_{s\downarrow 0}\frac{U((1-s)m+sm')-U(m)}{s}=\int_{\mathbb{T}^d}\frac{\delta U}{\delta m}(m,y)\mathrm{d}(m-m')(y)
    \]
\end{definition}
\begin{remark}
    在该定义下,同样有
    \[
        U(m')-U(m)=\int_{0}^{1}\int_{\mathbb{T}^d} \int_{\mathbb{T}^d}\frac{\delta U}{\delta m}((1-s)m+sm',y)\mathrm{d}(m-m')(y)\mathrm{d}s,m,m'\in \mathscr{P}(\mathbb{T}^d),
    \]
    这一点与之前定义的线性泛函导数是一致的.
\end{remark}
\begin{definition}
    设$U:\mathscr{P}(\mathbb{T}^d)\to \mathbb{R}$是$C^1$的,且$\frac{\delta U}{\delta m}$关于第二个变量是连续可微的,则定义$U$的内蕴导数为
    \[
        D_m U(m,y)=\nabla_y \frac{\delta U}{\delta m}(m,y):\mathscr{P}(\mathbb{T}^d)\times \mathbb{T}^d\to \mathbb{R}^d
    \]
\end{definition}
\begin{remark}
    可以证明,该内蕴导数的定义与之前也是一致的.实际上,在上一节已经指出:在一定条件下,内蕴导数就是L导数,线性泛函导数就是外在导数,而L导数又是
    外在导数关于空间变量的梯度,因此这些不同定义可以看作用不同性质刻画"同一个对象",所以这些一致性的保持是自然的.
\end{remark}

依照同样的观点,在一定条件下,当固定空间变量$y$时,可以对线性泛函导数$\frac{\delta U}{\delta m}$再求一次线性泛函导数,这样
就可以归纳定义高阶线性泛函导数和高阶内蕴导数.这里只考虑二阶的情形.\par
$\frac{\delta^2 U}{\delta m^2}$是一个从$\mathscr{P}(\mathbb{T}^d)\times \mathbb{T}^d\times\mathbb{T}^d$到$\mathbb{R}$的连续映射,而
$D_{mm}^2 U(m,y,y')=D_{y,y'}^2 \frac{\delta^2 U}{\delta m^2}(m,y,y')$则是从$\mathscr{P}(\mathbb{T}^d)\times \mathbb{T}^d\times\mathbb{T}^d$到$\mathbb{R}^{d\times d}$的映射.
假设$U$的一阶和二阶线性泛函导数关于空间变量都是连续可微的,二阶线性泛函导数和$D_y \frac{\delta^2 U}{\delta m^2}$关于所有变量联合连续,则有以下关系成立:\par
(1)\ $\frac{\delta^2 U}{\delta m^2}(m,y,y')-\frac{\delta^2 U}{\delta m^2}(m,y)=\frac{\delta U}{\delta m}(m,y',y)-\frac{\delta U}{\delta m}(m,y')$;\par
(2)\ $D_y\frac{\delta^2 U}{\delta m^2}(m,y,y')=\frac{\delta}{\delta m}(D_m(m,y))(y')$;\par
(3)\ $D_m(D_mU(\cdot,y))(m,y')=D^2_{mm}U(m,y,y')$.\par
需要说明的是,上面的第三个等式并不是"天然"成立的,因为二阶内蕴导数并不是归纳定义得到的.

设$(B_t),(W_t)$是一个完备带流概率空间上的两个布朗运动.考虑方程
\[
    dX_t=-\beta_t(X_t)\mathrm{d}t+\sqrt{2}(\mathrm{d}B_t+\mathrm{d}W_t),t\in[0,T],
\]
$(m_t)=(\mathscr{L}(X_t))$.映射$U:[0,T]\times \mathbb{T}^d\times \mathscr{P}(\mathbb{T}^d)$满足一系列正则性条件(具体见\citep{pfjp}的定义2.4.4),则存在一族随机变量
$(\epsilon_{s,t})_{t\in [0,T],s\leq t}$,使得如下局部Ito-Taylor展开成立:
\begin{align*}
&\frac{1}{h}\biggl(\mathbb{E}\bigl[U\bigl(t + h, x + \sqrt{2}(W_{t+h}), m_{t+h}\bigr) - U\bigl(t + h, x + \sqrt{2}(W_{t+h} ), m_t\bigr) \bigm| \mathcal{F}_t\bigr]\biggr) \\
=& \Delta_x U\bigl(t, x + \sqrt{2}(W_t), m_t\bigr) \\
&\quad + 2\int_{\mathbb{T}^d} \mathrm{div}_y\bigl[D_m U\bigr]\bigl(t, x + \sqrt{2}(W_t ), m_t, y\bigr) dm_t(y) \\
&\quad - \int_{\mathbb{T}^d} D_m U\bigl(t, x + \sqrt{2}(W_t ), m_t, y\bigr) \cdot \beta_t(y) dm_t(y) \\
&\quad + 2\int_{\mathbb{T}^d} \mathrm{div}_x D_m U \bigl(t, x + \sqrt{2}(W_t ), m_t, y\bigr) dm_t(y) \\
&\quad + \int_{[\mathbb{T}^d]^2} \mathrm{tr}\biggl(D_{mm}^2 U\bigl(t, x + \sqrt{2}(W_t ), m_t, y, y'\bigr)\biggr) dm_t(y) dm_t(y') \\
&\quad + \epsilon_{t,t+h}.
\end{align*}\par
上述公式的完整陈述及证明见\citep{pfjp}的附录A.3.
\section{若干例子}
对上一节定义的几种导数,我们介绍几个简单可计算的例子,并体现它们之间的联系.\par
称函数$f:\mathscr{M}_p\to \mathbb{R}$为$C^1_b-$柱函数或$f\in \mathscr{F}C^1_b(\mathscr{M}_p)$,是指:存在$n\in \mathbb{N}$,$g\in C^2(\mathbb{R}^n)$和
${h_i:1\leq i\leq n}\subset C^1_b(\mathbb{R}^d)$,使得
\[
    f(\mu)=g(\mu(h_1),\cdots,\mu(h_n)),\mu\in \mathscr{P}_p.
\]
柱函数是应用较广的一类函数,例如取$n=1,g(x)=x$以及适当的$h:\mathbb{R}^d\to \mathbb{R}$,此时$f(\mu)=\mu(h)$就是一个线性函数,
当$\mu\in \mathscr{M}_p$或$\mathscr{P}_p$时,容易计算
\begin{equation}
    \begin{aligned}
        &D^E f(\mu)(x)=h(x),\\
        &\tilde{D}^E f(x)=h(x)-\mu(h),
    \end{aligned}
\end{equation}
并且可以验证$D^E f$就是线性泛函导数,且根据第三章计算知,当$h$连续可微时,$D^L f(\mu)(x)=\nabla h(x)$.\par
对于任意光滑函数$g:\mathbb{R}\to \mathbb{R}$,
\begin{equation}
    \begin{aligned}
        &f(\mu+s\delta_x)-f(\mu)\\
        =&g((\mu+s\delta_x)(h))-g(\mu(h))\\
        =&\int_{0}^{1} g'(\mu(h)+t((\mu+s\delta_x)(h)-\mu(h)))((\mu+s\delta_x)(h)-\mu(h))\mathrm{d}t\\
        =&\int_{0}^{1} g'(\mu(h)+sth(x))sh(x)\mathrm{d}t,
    \end{aligned}
\end{equation}
观察并验证可得,$D^E f(\mu)(x)=g'(\mu(h))h(x)$.
对于$f$的内蕴导数,取光滑向量场$\phi$,
\begin{equation}
    \begin{aligned}
        &f(\mu\circ (\mathrm{id}+s\phi)^{-1})-f(\mu)\\
        =&g(\mu(h\circ (\mathrm{id}+s\phi)))-g(\mu(h))\\
        =&\int_{0}^{1} g'(\mu(h)+t(\mu(h\circ (\mathrm{id}+s\phi))-\mu(h)))(\mu(h\circ (\mathrm{id}+s\phi))-\mu(h)))\mathrm{d}t,
    \end{aligned}
\end{equation}
设$h$是紧支光滑的,则$\lim_{s\downarrow 0}\frac{h(x+s\phi(x))-h(x)}{s}=\ipr{\nabla h(x)}{\phi(x)}$,观察并验证可得
$D^I_\phi f(\mu)=g'(\mu(h))\ipr{\nabla}{\phi}$,也即$D^I f(\mu)=g'(\mu(h))\nabla h$.
以上讨论均可推至一般的柱函数$f(\mu)=g(\mu(h_1),\cdots,\mu(h_n))$,
\begin{equation}
    \begin{aligned}
        &D^E f(\mu)=\sum_{i=1}^{n}\partial_{i}g(\mu(h_1),\cdots,\mu(h_n))h_i,\\
        &D^L f(\mu)=\sum_{i=1}^{n}\partial_{i}g(\mu(h_1),\cdots,\mu(h_n))\nabla h_i.
    \end{aligned}
\end{equation}

下面考虑卷积函数$f(\mu)=\mu(h*\mu)=\int_{\mathbb{R}^d} (h*\mu)(x) \mathrm{d}x=\int_{\mathbb{R}^d}\int_{\mathbb{R}^d}h(x-y)\mathrm{d}\mu(y)\mathrm{d}\mu(x)$,则
$f(\mu+s\delta_z)=f(\mu)+s\mu(h)+s((h+\bar{h})*\mu)(z)+s^2 h(0)$,其中$\bar{h}(x)=h(-x)$,即$D^E f(\mu)(z)=((h+\bar{h})*\mu)(z)$,这也与第三章的计算相对应.
